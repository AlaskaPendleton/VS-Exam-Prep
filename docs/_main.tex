% Options for packages loaded elsewhere
\PassOptionsToPackage{unicode}{hyperref}
\PassOptionsToPackage{hyphens}{url}
%
\documentclass[
]{book}
\title{Vascular Surgery Board Review}
\author{Editors: Adam Johnson, MD, MPH; Matt Smith, MD, PhD; and Audible Bleeding}
\date{2022-02-12}

\usepackage{amsmath,amssymb}
\usepackage{lmodern}
\usepackage{iftex}
\ifPDFTeX
  \usepackage[T1]{fontenc}
  \usepackage[utf8]{inputenc}
  \usepackage{textcomp} % provide euro and other symbols
\else % if luatex or xetex
  \usepackage{unicode-math}
  \defaultfontfeatures{Scale=MatchLowercase}
  \defaultfontfeatures[\rmfamily]{Ligatures=TeX,Scale=1}
\fi
% Use upquote if available, for straight quotes in verbatim environments
\IfFileExists{upquote.sty}{\usepackage{upquote}}{}
\IfFileExists{microtype.sty}{% use microtype if available
  \usepackage[]{microtype}
  \UseMicrotypeSet[protrusion]{basicmath} % disable protrusion for tt fonts
}{}
\makeatletter
\@ifundefined{KOMAClassName}{% if non-KOMA class
  \IfFileExists{parskip.sty}{%
    \usepackage{parskip}
  }{% else
    \setlength{\parindent}{0pt}
    \setlength{\parskip}{6pt plus 2pt minus 1pt}}
}{% if KOMA class
  \KOMAoptions{parskip=half}}
\makeatother
\usepackage{xcolor}
\IfFileExists{xurl.sty}{\usepackage{xurl}}{} % add URL line breaks if available
\IfFileExists{bookmark.sty}{\usepackage{bookmark}}{\usepackage{hyperref}}
\hypersetup{
  pdftitle={Vascular Surgery Board Review},
  pdfauthor={Editors: Adam Johnson, MD, MPH; Matt Smith, MD, PhD; and Audible Bleeding},
  hidelinks,
  pdfcreator={LaTeX via pandoc}}
\urlstyle{same} % disable monospaced font for URLs
\usepackage{longtable,booktabs,array}
\usepackage{calc} % for calculating minipage widths
% Correct order of tables after \paragraph or \subparagraph
\usepackage{etoolbox}
\makeatletter
\patchcmd\longtable{\par}{\if@noskipsec\mbox{}\fi\par}{}{}
\makeatother
% Allow footnotes in longtable head/foot
\IfFileExists{footnotehyper.sty}{\usepackage{footnotehyper}}{\usepackage{footnote}}
\makesavenoteenv{longtable}
\usepackage{graphicx}
\makeatletter
\def\maxwidth{\ifdim\Gin@nat@width>\linewidth\linewidth\else\Gin@nat@width\fi}
\def\maxheight{\ifdim\Gin@nat@height>\textheight\textheight\else\Gin@nat@height\fi}
\makeatother
% Scale images if necessary, so that they will not overflow the page
% margins by default, and it is still possible to overwrite the defaults
% using explicit options in \includegraphics[width, height, ...]{}
\setkeys{Gin}{width=\maxwidth,height=\maxheight,keepaspectratio}
% Set default figure placement to htbp
\makeatletter
\def\fps@figure{htbp}
\makeatother
\setlength{\emergencystretch}{3em} % prevent overfull lines
\providecommand{\tightlist}{%
  \setlength{\itemsep}{0pt}\setlength{\parskip}{0pt}}
\setcounter{secnumdepth}{5}
\usepackage{booktabs}
\ifLuaTeX
  \usepackage{selnolig}  % disable illegal ligatures
\fi
\usepackage[]{natbib}
\bibliographystyle{plainnat}

\begin{document}
\maketitle

{
\setcounter{tocdepth}{1}
\tableofcontents
}
\hypertarget{about}{%
\chapter{About}\label{about}}

The content was developed here by the \href{https://www.audiblebleeding.com/about-1/}{Audible Bleeding Team} to accompany our board review podcast episodes.

\hypertarget{usage}{%
\section{Usage}\label{usage}}

This is not a comprehensive textbook but instead an outline of the most high yield information to help guide board preparation.

\hypertarget{comments-questions-or-contributions}{%
\section{Comments, Questions or Contributions}\label{comments-questions-or-contributions}}

Please visit our \href{https://github.com/adam-mdmph/VS-Board-Review}{github page} or \href{mailto:audiblebleeding@vascularsociety.org}{send us an email}.

\hypertarget{cerebrovascular}{%
\chapter{Cerebrovascular}\label{cerebrovascular}}

\textbf{07 Jan 2019:} \emph{Adam Johnson, MD, MPH; Nicole Rich, MD, MPH; Kevin
Kniery, MD, MPH}

\hypertarget{available-guidelines}{%
\section{Available Guidelines}\label{available-guidelines}}

\href{https://www.jvascsurg.org/article/S0741-5214(21)00893-4/fulltext}{Society for Vascular Surgery clinical practice guidelines for
management of extracranial cerebrovascular
disease}
\citep{aburahmaSocietyVascularSurgery2022}

\hypertarget{presentation-and-diagnosis}{%
\section{Presentation and Diagnosis}\label{presentation-and-diagnosis}}

\begin{enumerate}
\def\labelenumi{\arabic{enumi}.}
\item
  \textbf{What is the definition of crescendo TIAs?}

  Frequent repetitive neurological attacks without complete resolution
  of the deficit between the episodes, producing the same deficit but
  no progressive deterioration in neurological function If a
  progressive deterioration then it is a stroke in evolution.
\item
  \textbf{Who needs to be screened?}

  Only 15\% of stroke victims have a warning TIA before a stroke so
  waiting until symptoms occur is not ideal. The purpose of carotid
  bifurcation imaging is to detect ``stroke-prone'' carotid bifurcation
  plaque and identify a high-risk patient likely to benefit from
  therapy designed to reduce stroke risk.

  The absence of a neck bruit does not exclude the possibility of a
  significant carotid bifurcation lesion - focal ipsilateral carotid
  bruits in symptomatic patients has a sensitivity of 63\% and a
  specificity of 61\% for high-grade carotid stenosis (range, 70\%-99\%).

  Screening of the general population is not indicated. Screening
  should be considered for patients with:

  \begin{itemize}
  \item
    Evidence of clinically significant peripheral vascular disease
    regardless of age
  \item
    Patients aged \textgreater65 years with a history of one or more of the
    following atherosclerotic risk factors:

    \begin{itemize}
    \item
      CAD
    \item
      Smoking
    \item
      Hypercholesterolemia
    \end{itemize}
  \item
    In general, the more risk factors present, the higher the yield
    of screening should be expected.
  \item
    The benefit of prophylactic treatment of high grade stenosis is
    estimated at a 1-2\% stroke reduction risk per year.
    \citep{naylorWhyManagementAsymptomatic2015}
  \item
    Keep in mind that intervention (CEA/CAS) has only demonstrated a
    benefit in asymptomatic patient with life expectancy greater
    than 3 years. \citep{bulbuliaAsymptomaticCarotidSurgery2017, halliday10yearStrokePrevention2010, rosenfieldRandomizedTrialStent2016}
  \end{itemize}
\item
  \textbf{US findings that confirm disease}

  \begin{itemize}
  \item
    50-69\% stenosis of ICA - Low sensitivity for 50-69\% stenosis - a
    negative ultrasound in symptomatic patients necessitates
    additional imaging

    \begin{itemize}
    \tightlist
    \item
      PSV 125-229 cm/sec
    \item
      EDV 40-100
    \item
      Internal/Common Carotid PSV Ratio 2-4
    \end{itemize}
  \item
    70-99\% stenosis of ICA

    \begin{itemize}
    \item
      PSV \textgreater/= 230 cm/sec
    \item
      EDV \textgreater100 (EDV \textgreater{} 140 cm/sec most sensitive for stenosis \textgreater80\%)
    \item
      Internal/Common Carotid PSV Ratio \textgreater{} 4
    \end{itemize}
  \item
    Velocity-based estimation of carotid artery stenosis may need to
    be adjusted in certain circumstances

    \begin{itemize}
    \item
      Higher velocities in women than in men
    \item
      Higher velocities in the presence of contralateral carotid
      artery occlusion.
    \end{itemize}
  \item
    High carotid bifurcation, severe arterial tortuosity, extensive
    vascular calcification, and obesity may also reduce the accuracy
    of DUS imaging
  \end{itemize}
\item
  \textbf{Other Imaging Modalities}

  \begin{itemize}
  \item
    CTA

    \begin{itemize}
    \item
      Pro - fast, sub-millimeter spatial resolution, visualize
      surrounding structures
    \item
      Con - cost, contrast exposure
    \end{itemize}
  \item
    MRA

    \begin{itemize}
    \item
      Pro - no contrast administered; analyze plaque morphology
    \item
      Con - Does not visualize calcium in plaque; overestimates
      the degree of stenosis (False positive for 50-69\% to be read
      as \textgreater70\%)
    \end{itemize}
  \item
    Catheter-based digital subtraction imaging (DSA)

    \begin{itemize}
    \item
      Still considered by many the gold-standard imaging modality
    \item
      Reserved for individuals with conflicting less-invasive
      imaging or those considered for CAS
    \item
      Con - cost and risk of stroke
    \end{itemize}
  \end{itemize}
\end{enumerate}

\hypertarget{management}{%
\section{Management}\label{management}}

\hypertarget{optimal-medical-therapy}{%
\subsection{\texorpdfstring{\textbf{Optimal medical therapy}}{Optimal medical therapy}}\label{optimal-medical-therapy}}

\textbf{Hypertension}

\begin{itemize}
\item
  Lowering blood pressure to a target \textless140/90 mmHg by lifestyle
  interventions and anti-hypertensive treatment is recommended in
  individuals who have hypertension with asymptomatic carotid
  atherosclerosis or those with TIA or stroke after the hyper-acute
  period.
\item
  Each 10-mm Hg reduction in blood pressure among hypertensive
  patients decreases the risk for stroke by 33\%.
\end{itemize}

\textbf{Diabetes}

\begin{itemize}
\tightlist
\item
  Glucose control to nearly normoglycemic levels (target hemoglobin
  A1C \textless7\%) is recommended among diabetic patients to reduce
  microvascular complications and, with lesser certainty,
  macrovascular complications other than stroke.
\end{itemize}

\textbf{Lipid abnormalities}

\begin{itemize}
\item
  Risk of stroke decreased by \textgreater15\% for every 10\% reduction in serum
  LDL in patients with known coronary or other atherosclerosis
\item
  Statin agents are recommended targeting LDL of 100 mg/dL, for those
  with coronary heart disease or symptomatic atherosclerotic disease,
  and LDL of 70 mg/dL for very high-risk persons with multiple risk
  factors
\item
  High dose statin therapy in patients with TIA/stroke reduce future
  rates of stroke or cardiovascular events but not overall mortality
  at 5 years. \citep{karamHighDoseAtorvastatinStroke2008}
\end{itemize}

\textbf{Smoking} - Physician counseling is an important and effective
intervention that reduces smoking in patients by 10\% to 20\%

\textbf{Antithrombotic therapy} - There is no evidence to suggest that
antiplatelet agents other than aspirin have improved benefit in
asymptomatic patients with carotid atherosclerosis

\hypertarget{carotid-endarterectomy}{%
\subsection{\texorpdfstring{\textbf{Carotid endarterectomy}}{Carotid endarterectomy}}\label{carotid-endarterectomy}}

\textbf{Timing}

\begin{itemize}
\item
  Recommendations on when to operate after a stroke

  \begin{itemize}
  \item
    Acute stroke with a fixed neurologic deficit of \textgreater6h duration -
    When the patient is medically stable, treatment in less than or
    equal to 2 weeks after the stroke is preferable.
    \citep{rothwellEndarterectomySymptomaticCarotid2004, meershoekTimingCarotidIntervention2018}
  \item
    Consider urgent intervention in a medically stable patient with
    mild-moderate neurologic deficit, if there is a significant area
    of ischemic penumbra at risk for progression
  \item
    Stroke in evolution (fluctuating / evolving neuro deficit) or
    crescendo TIA (repetitive transient ischemia w improvement
    between events) - If neuro status is not stabilized by medical
    intervention consider urgent CEA
  \item
    CEA is preferred to CAS based on an increased embolic potential
    of carotid lesions that present in this fashion.
    \citep{rantnerEarlyEndarterectomyCarries2017}
  \item
    Management of acute stroke \citep{powers2018GuidelinesEarly2018}

    \begin{itemize}
    \item
      \textless4.5hrs from onset of symptoms - tPA unless
      contraindication

      \begin{itemize}
      \item
        Age \textgreater80 and diabetes are contraindication to tPA after
        3hrs.
      \item
        Other contraindications - high BP, intracranial
        hemorrhage, recent stroke or head trauma, spine/brain
        surgery within 3mo, GI bleed within 21d
      \end{itemize}
    \item
      \textless6hr from onset of symptoms - catheter directed therapy
    \end{itemize}
  \end{itemize}
\item
  What is the only emergent indication for CEA?

  \begin{itemize}
  \tightlist
  \item
    Crescendo TIAs or a stroke in evolution with a surgically
    correctable lesion that is identified
  \end{itemize}
\end{itemize}

\textbf{Intraoperative Techniques}

\begin{itemize}
\item
  General concepts

  \begin{itemize}
  \tightlist
  \item
    Patch angioplasty or eversion endarterectomy are recommended
    rather than primary closure to reduce the early and late
    complications of CEA (GRADE 1, Level of Evidence A).
  \end{itemize}
\item
  Neuromonitoring/Shunting options during a carotid endarterectomy

  \begin{itemize}
  \item
    Local anesthesia with direct neuro monitoring - the patient is
    awake and moving to command throughout the case. Though improved
    neuromonitoring has not been shown to reduce MI rate with CEA
  \item
    Stump pressure Clamp the inflow and place butterfly attached to
    a-line tubing into the internal carotid If stump pressure is \textgreater{}
    40 mmHg can proceed, if \textless{} 40 place shunt
  \item
    EEG Neuromonitoring - EEG tech places neuromonitoring, monitored
    by intraop tech and neurologist remotely, generally clamp ICA
    for 3 minutes before proceeding, if any deficits unclamp, await
    normalization of EEG then proceed
  \item
    Non-selective shunting - shunt all carotids
  \end{itemize}
\item
  Techniques to reach internal carotid lesions that are high?

  \begin{itemize}
  \item
    Nasotracheal intubation will help extend the neck to reach
    higher lesions
  \item
    Divide posterior belly of digastric to reach high lesions with
    care to watch for glossopharyngeal
  \item
    Styloidectomy
  \item
    Mandible subluxation with assistance from ENT if previous
    techniques fail.
  \end{itemize}
\item
  What is the best technique for a patient with a kinked internal
  carotid artery?

  \begin{itemize}
  \item
    Eversion carotid endarterectomy will allow you to reduce the
    redundancy
  \item
    Otherwise, no advantage has been shown between eversion or
    patch, both can be shunted
  \end{itemize}
\item
  Discuss nerve injuries -- where you would encounter these and what
  deficit would be seen

  \begin{itemize}
  \item
    Hypoglossal Just above the bifurcation of the carotid artery
    Will see tongue deviation to the side of injury
  \item
    Glossopharyngeal High dissections under digastric Difficulty
    swallowing, aspiration risk, can be devastating
  \item
    Vagus Adjacent and lateral to carotid, injury occurs with
    carotid clamping, Hoarseness is noted as RLN is a branch off of
    vagus
  \item
    Marginal Mandibular (Off of facial nerve) Retraction at the
    angle of the jaw for high dissections Leads to the corner of lip
    drooping, can be confused with a neuro deficit following the
    case
  \end{itemize}
\end{itemize}

\textbf{Postoperative Complications}

\begin{itemize}
\item
  What to do if neuro deficits following your carotid endarterectomy

  \begin{itemize}
  \item
    If in OR -- perform duplex, if normal open wound and shoot
    cerebral angiogram
  \item
    If in Recovery or on the floor -- many would consider CTA first
    vs duplex to look for thrombosis
  \end{itemize}
\item
  Risk factors and how to manage hyperperfusion syndrome?

  \begin{itemize}
  \item
    Defined as an ipsilateral headache, hypertension, seizures, and
    focal neurological deficits can present 2-3 days out from
    surgery
  \item
    Patients with uncontrolled hypertension are at risk for
    hyperperfusion syndrome, clinical practice guidelines by SVS
    recommend strict BP control following CEA, maintain a pressure
    less than 140/80
  \end{itemize}
\item
  High risk groups

  \begin{itemize}
  \tightlist
  \item
    ESRD patients have higher rates of perioperative stroke, but
    also have higher rates of stroke if not revascularized.
    \citep{klarinPerioperativeLongtermImpact2016}
  \end{itemize}
\end{itemize}

\textbf{Long term complications and follow up}

\begin{itemize}
\item
  Recommend f/u US at \textless/=30 days. \textgreater/= 50\% stenosis requires further
  imaging.
\item
  Contralateral stenosis

  \begin{itemize}
  \item
    The risk of progression for moderate stenosis at the initial
    surveillance to severe stenosis can be as high as five times
  \item
    Requires post-operative surveillance.
  \end{itemize}
\end{itemize}

\hypertarget{carotid-artery-stenting}{%
\subsection{Carotid Artery Stenting}\label{carotid-artery-stenting}}

\begin{itemize}
\item
  In patients aged \textgreater70 undergoing CAS the risk of stroke was the
  highest, presumably due to calcific disease in the arch

  \begin{itemize}
  \item
    Lesion-specific characteristics are thought to increase the risk
    of cerebral vascular events after CAS and include a ``soft''
    lipid-rich plaque identified on noninvasive imaging, extensive
    (15 mm or more) disease, a pre-occlusive lesion, and
    circumferential heavy calcification
  \item
    This can be reduced, but not eliminated, by using flow-reversal
    embolic protection rather than distal filter protection
  \end{itemize}
\item
  Limited data on CAS in asymptomatic patients - currently is not
  supported by guidelines or considered reimbursable
\item
  Consider CAS in symptomatic patients with \textgreater50\% stenosis who are poor
  candidates for CEA due to severe uncorrectable medical comorbidities
  and/or anatomic considerations

  \begin{itemize}
  \item
    Ipsilateral neck dissection or XRT - equivalent periprocedural
    stroke rate to CEA, but increased later stroke rate. CEA higher
    rates of cranial nerve damage (9\%).
    \citep{giannopoulosRevascularizationRadiationinducedCarotid2018}
  \item
    Contralateral vocal cord paralysis
  \item
    Lesions that extend proximally to the clavicle or distal to C2
  \end{itemize}
\item
  Transfemoral Approach vs Transcarotid approach

  \begin{itemize}
  \tightlist
  \item
    ROADSTER Trial - single arm study with flow reversal for
    cerebral protection. Suggest lower rates of post-op stroke
  \end{itemize}
\item
  Post-op follow up - Dual-platelet therapy should be continued for 1
  month after the procedure, and aspirin should be continued
  indefinitely

  \begin{itemize}
  \tightlist
  \item
    In stent restenosis (\textgreater50\%) - repeat angioplasty or stent have
    low incidence of periprocedural stroke but failed to improve
    long term stroke/death/MI or patency rates.
    \citep{chungPercutaneousInterventionCarotid2016a}
  \end{itemize}
\end{itemize}

\hypertarget{management-of-uncommon-disease-presentations}{%
\subsection{Management of uncommon disease presentations}\label{management-of-uncommon-disease-presentations}}

\begin{itemize}
\item
  Occluded Carotid What to do for occluded carotid?

  \begin{itemize}
  \tightlist
  \item
    Leave it alone
  \end{itemize}
\item
  What if occluded carotid is still causing TIAs?

  \begin{itemize}
  \item
    External carotid endarterectomy and ligation of internal
  \item
    The addition of oral anticoagulation is likely to reduce the
    rate of recurrent CVA
  \end{itemize}
\item
  What if the patient has severe vertebrobasilar insufficiency and
  carotid artery disease?

  \begin{itemize}
  \tightlist
  \item
    Should undergo carotid revascularization first to improve flow
  \item
    Vertebrobasilar insufficiency characterized by dizziness,
    ataxia, nausea, vertigo and bilateral weakness.
    \citep{limanetoPathophysiologyDiagnosisVertebrobasilar2017}
  \end{itemize}
\item
  What about tandem lesions in the carotid in a symptomatic patient,
  carotid bulb and carotid siphon lesion (high ICA)? How should you
  treat this?

  \begin{itemize}
  \tightlist
  \item
    Treat carotid bulb first, likely the embolic source
  \end{itemize}
\item
  Carotid artery dissection

  \begin{itemize}
  \item
    Patients with carotid dissection should be initially treated
    with antithrombotic therapy (antiplatelet agents or
    anticoagulation) (GRADE 1, Level of Evidence C).
  \item
    Indications for endovascular treatment of carotid artery
    dissection \citep{cohenSinglecenterExperienceEndovascular2012, markusAntiplateletTherapyVs2019a, phamEndovascularStentingExtracranial2011}

    \begin{itemize}
    \item
      Ongoing symptoms on best medical therapy
    \item
      Contraindication to antithrombotics
    \item
      Pseudoaneurysm
    \end{itemize}
  \end{itemize}
\item
  Simultaneous coronary and carotid disease

  \begin{itemize}
  \item
    Patients with symptomatic carotid stenosis will benefit from CEA
    before or concomitant with CABG. The timing of the intervention
    depends on the clinical presentation and institutional
    experience (GRADE 1, Level of Evidence B).
  \item
    Patients with severe bilateral asymptomatic carotid stenosis,
    including stenosis and contralateral occlusion, should be
    considered for CEA before or concomitant with CABG (GRADE 2,
    Level of Evidence B)
  \item
    Patients undergoing simultaneous CEA/CABG demonstrate highest
    mortality. \citep{naylorSystematicReviewOutcomes2003}
  \end{itemize}
\end{itemize}

\hypertarget{prospective-trials---must-reads}{%
\section{Prospective Trials - MUST READS}\label{prospective-trials---must-reads}}

\begin{enumerate}
\def\labelenumi{\arabic{enumi}.}
\item
  Asymptomatic Carotid Atherosclerosis Study (ACAS)

  \begin{itemize}
  \item
    Compared medical management with CEA in asymptomatic patients
    with \textgreater{} 60\% stenosis
  \item
    5-year stroke and death rate was 5.1\% vs 11\%
  \item
    In women, the benefit of CEA was not as certain as 5y stroke and
    death rates were 7.3\% vs.~8.7\%
  \item
    This was pre statin and clopidogrel era
  \end{itemize}
\item
  North American Symptomatic Carotid Endarterectomy Trial (NASCET)
  \citep{northamericansymptomaticcarotidendarterectomytrialcollaboratorsBeneficialEffectCarotid1991}

  \begin{itemize}
  \item
    Compared medical management vs CEA for symptomatic patients with
    moderate (50-69\%) and severe stenosis (\textgreater70\%)
  \item
    Only moderate impact for patients with moderate stenosis
    (50-69\%)
  \item
    Symptomatic patients with \textgreater70 \% stenosis benefited from CEA, at
    18 months 7\% major stroke in surgical arm, and a 24\% stroke rate
    in medical arm. 29\% reduction in 5-year risk of stroke or death

    \begin{itemize}
    \tightlist
    \item
      Patients with severe \textgreater70\% stenosis had such a dramatic
      effect the trial was stopped early for this subset and all
      referred for endarterectomy
    \end{itemize}
  \item
    No benefit is shown in symptomatic patients with \textless{} 50\% stenosis
  \item
    European studies have shown similar results

    \begin{itemize}
    \item
      ACST = ACAS
    \item
      ECST = NASCET.
    \end{itemize}
  \end{itemize}
\item
  Carotid Revascularization Endarterectomy versus Stenting Trial
  (CREST)

  \begin{itemize}
  \item
    Compared CEA vs.~CAS in both symptomatic and asymptomatic
    patients.
  \item
    Composite endpoint of 30-day stroke, MI, death equivalent
    between CEA and CAS
  \item
    CAS had a significantly higher incidence of stroke and death
    than CEA and CEA higher incidence of MI

    \begin{itemize}
    \tightlist
    \item
      Follow up at 10 years demonstrated no difference in
      composite stroke/MI/death but increased rate of stroke/death
      in stented patients likely attributable to increased
      periprocedural stroke. \citep{brottLongTermResultsStenting2016b}
    \end{itemize}
  \item
    Subanalyses identified that older patients (\textgreater70y) had better
    outcomes after CEA than CAS, the QOL impact of stroke was more
    significant than that of MI, and anatomic characteristics of
    carotid lesions (longer, sequential, remote) were predictive of
    increased stroke and death after CAS
  \item
    Unfortunately, this study provides a benchmark to strive for,
    but no other large trials have achieved these results.
  \end{itemize}
\item
  ROADSTER

  \begin{itemize}
  \item
    Single arm feasibility trial of transcarotid carotid stenting
  \item
    The results of the ROADSTER trial demonstrate that the use of
    the ENROUTE Transcarotid NPS is safe and effective at preventing
    stroke during CAS. The overall stroke rate of 1.4\% is the lowest
    reported to date for any prospective, multicenter clinical trial
    of CAS.
  \end{itemize}
\item
  Trials to look out for in the next few years

  \begin{itemize}
  \item
    CREST-2 - multicenter, randomized controlled trial is underway
    that is evaluating revascularization against modern intensive
    medical management
  \item
    ACT-1 and ACST-2- the role of intervention in asymptomatic
    patients, designed to compare the early and long-term results of
    CEA vs CAS and best medical management
  \item
    ROADSTER-2 - TCAR
  \end{itemize}
\end{enumerate}

\hypertarget{upper-extremity-and-thoracic-outlet}{%
\chapter{Upper Extremity and Thoracic Outlet}\label{upper-extremity-and-thoracic-outlet}}

\textbf{21 Jan 2021}: \emph{Kush Sharma, MD and Ashraf Mansour, MD}

\hypertarget{anatomy-exposure-of-vessels}{%
\section{Anatomy/ Exposure of Vessels}\label{anatomy-exposure-of-vessels}}

\textbf{What are the zones of the upper extremity?}
\citep{illig57UpperExtremity2019, illig57UpperExtremity2019}

Division of the upper extremity into three zones:

\begin{enumerate}
\def\labelenumi{\arabic{enumi}.}
\item
  Intrathoracic zone including aortic arch, innominate artery,
  subclavian artery bilaterally, innominate veins, and SVC
\item
  Thoracic outlet (base of neck to the axilla including the
  subclavian, proximal vertebral, proximal axillary arteries/veins)
\item
  Axilla to fingers (the arm)
\end{enumerate}

\textbf{What are some common exposures for major upper extremity arteries?}

Right Subclavian Artery: Medial sternotomy (proximal) or right
supraclavicular area (mid/distal)

Left Subclavian Artery: Anterolateral thoracotomy in emergent setting
for proximal left subclavian artery control. When third space
sternotomy, supraclavicular incision with thoracotomy ``trap door''
exposure

Supraclavicular incision: After division of the platysma and clavicular
head of the SCM, fat pat of varying thickness contains the omohyoid
muscle. This should be divided and placed superiorly/laterally. At this
point, the anterior scalene muscle is exposed medially with phrenic
nerve running in lateral to medial direction. Division of anterior
scalene for carotid/subclavian bypass should be performed as close to
the first rib as possible. After this is performed, the subclavian
artery is exposed.

Axillary Artery: Infraclavicular exposure below middle 1/3rd of
clavicle. Pec major split and pec minor freed at lateral wound. Axillary
vein followed by deep and superior to get to artery

Anatomically bound by the first rib proximally and the lateral edge of
the teres major muscle distally. For exposure of the first part of the
axillary artery, the ipsilateral arm is abducted approximately 90
degrees and horizontal skin incision 2 cm below the middle third of the
clavicle. Underlying pec major is split by bluntly separating the fibers
and followed by exposing the tough clavipectoral fascia. At the lateral
wound, the pec minor can be freed and laterally retracted. The axillary
vein is first structure encountered in the sheath and the artery lies
just superior and deep to the vein. Make sure to avoid nerves of
brachial plexus that lie deep to first part of axillary artery and are
at risk for injury during blind placement of occluding arterial clamps.
\citep{garygwindAnatomicExposuresVascular2013}

\textbf{What steps are involved for brachial artery exposure?}

Brachial artery: incision between biceps/triceps on medial arm (avoid
basilic vein damage in subcutaneous and deep to the fasia at medial
biceps. Median nerve seen and retracted. Two brachial vein are paired
adjacent to artery.

Superficial location makes it vulnerable to injury and accounts for most
vascular injuries of upper extremities. Brachial artery exposure
involves a 5-8 cm longitudinal incision in the groove between the
biceps/triceps muscles on the medial aspect of the arm. In the lower
half of the arm, take care to avoid basilic vein damage in the
subcutaneous tissue. Neurovascular bundle exposed by incising the deep
fascia at the medial border of the biceps muscle, which is retracted
anteriorly. After retracting basilic vein into posterior wound ,brachial
sheath is opened and median nerve is most superficial structure and
retracted. The artery lies deep to the nerve and surrounded by two
brachial veins. Posteriorly, is the presence of the ulnar nerve.

Brachial Artery bifurcates at the radial tuberosity into radial/ulnar
branches. After the bifurcation and immediately after its origin, the
ulnar artery gives off a short common interosseous branch, which
bifurcates at the hiatus in the proximal interosseous membrane. Exposure
of brachial artery in the antecubital fossa requires a transverse skin
incision 1 cm distal to the midpoint of the antecubital crease. After
deepening, avoid injury to subcutaneous veins and mobilize the basilic
vein medially. Medial antebrachial cutaneous nerve should be protected.
Divide the bicipital aponeurosis and after division, exposure of the
brachial artery is present, which is flanked by two deep veins and
crossing branches. Isolation of brachial artery requires ligation and
division of these crossing vein branches.

Radial artery at the wrist with 2-3 cm longitudinal incision generally
between radial artery and cephalic vein. Radial artery was exposed by
incising the antebrachial fascia just medial to the radius. Two veins
accompany the artery and should be dissected away during arterial
isolation. The superficial radial nerve and its medial/lateral branches
course between the cephalic vein and radial artery in the area.

Exposure of the ulnar artery is by coursing beneath the superficial
flexor muscles in the proximal forearm, emerging near the ulnar border
at the point midway between the elbow and the wrist. In the distal
forearm, the ulnar artery course just beneath the antebrachial fascia
and is easily exposed through a longitudinal incision placed radial to
the flexor carbi ulnaris. The palmar branch of the ulnar nerve courses
the superficial to the antebrachial fascia and should be preserved
during arterial exposure

\textbf{What common aberrant upper extremity/arch anatomy is important to be
aware of?}

\begin{itemize}
\item
  Bovine arch with left common carotid/left subclavian have common
  origin
\item
  Vertebral artery directly off the aortic arch
\item
  Aberrant right subclavian where innominate becomes right CCA and
  right subclavian distal to last branch on left side passing behind
  esophagus to supply the right arm
\end{itemize}

\hypertarget{epidemiology-etiology-and-diagnostic-evaluation}{%
\section{Epidemiology, etiology, and diagnostic evaluation}\label{epidemiology-etiology-and-diagnostic-evaluation}}

\textbf{How does evaluation of upper extremity ischemia differentiate from
lower extremity ischemia?} \citep{shuja117UpperExtremity}

\begin{itemize}
\item
  Upper extremity ischemia \textless5\% of patients with limb ischemia and in
  contrast to lower extremity, atherosclerosis is not a major
  contributor to upper extremity ischemia
\item
  Vast majority of cases caused by autoimmune/connective tissue
  disorders
\end{itemize}

\textbf{How can upper extremity disease be classified?}

Anatomic Location:

\begin{itemize}
\tightlist
\item
  Large vs.~Small Vessel
\end{itemize}

Disease Process:

\begin{itemize}
\tightlist
\item
  Vasospastic or occlusive. Vasospastic disease is more responsive to
  pharmacologic management while occlusive requiring
  endovascular/surgical management.
\end{itemize}

\textbf{How should patients be evaluated who have concern for upper extremity
disease?}

Diagnostic Evaluation

\begin{enumerate}
\def\labelenumi{\arabic{enumi}.}
\item
  Detailed H+P evaluation (pulse palpation, auscultation at
  supraclavicular/infraclavicular fossa may reaveal a bruit concerning
  for subclavian artery stenosis, upper extremity neurovascular/skin
  exam)
\item
  Brachial/forearm blood pressures and if suspected claudication,
  measured at rest and 2-5 minutes after exercise. Look for a gradient
  of \textgreater20 mmHg is considered significant
\item
  Some or all of 6 P's of acute limb ischemia with symptoms occurring
  within 14 days are deemed acute
\item
  Doppler insonation of radial, ulnar, palmar, and digital arteries
\item
  Vascular Lab Evaluation

  \begin{enumerate}
  \def\labelenumii{\arabic{enumii}.}
  \item
    Segmental Pressure Measurements
  \item
    Duplex Ultrasound (look for large vessel occlusive disease)
  \end{enumerate}
\item
  Other Imaging

  \begin{enumerate}
  \def\labelenumii{\arabic{enumii}.}
  \tightlist
  \item
    CTA/MRA
  \end{enumerate}
\item
  Clinical Lab tests

  \begin{enumerate}
  \def\labelenumii{\arabic{enumii}.}
  \item
    Inflammatory disorders-CBC, ESR, ANA, RF
  \item
    Hypercoagulable screening
  \end{enumerate}
\end{enumerate}

\hypertarget{operationsprocedures}{%
\section{Operations/Procedures}\label{operationsprocedures}}

\textbf{What are some indications for carotid-subclavian bypass?}

\begin{enumerate}
\def\labelenumi{\arabic{enumi}.}
\item
  Atherosclerosis~
\item
  Staged revascularization prior to TEVAR for aneurysmal disease
  requiring coverage of the LSA
\end{enumerate}

\textbf{How does the exposure differentiate in transposition vs bypass?}

Exposure (Transposition vs Bypass)

\begin{itemize}
\item
  Arterial transposition via a short, transverse cervical incision
  above the clavicle between two heads of SCM (bypass is lateral to
  entire SCM)
\item
  Sub-platysmal flaps created and avoid EJ vein damage
\item
  Omohypoid divided between heads of SCM and IJ mobilized laterally
  (bypass IJ is mobilized medially to expose CCA and care must be
  taken to avoid phrenic nerve in more lateral approach)
\item
  CCA is reflected medially with vagus nerve~
\item
  On the left side, the thoracic duct is identifiable and divided
  followed by dividing the vertebral vein
\item
  Subclavian artery and proximal branches identified (anterior scalene
  is in lateral dissection)
\end{itemize}

\textbf{What are some common complications after carotid subclavian bypass in
order of highest to lowest incidence?}

Complications \citep{voigtOutcomesCarotidsubclavianBypass2019}

\begin{enumerate}
\def\labelenumi{\arabic{enumi}.}
\item
  Phrenic nerve palsy (most common) - most often managed
  conservatively.
\item
  Recurrent laryngeal palsy
\item
  Lymphatic leak
\item
  Neck hematoma
\end{enumerate}

\textbf{When carotid-subclavian bypass compared to transposition?}~~

\begin{enumerate}
\def\labelenumi{\arabic{enumi}.}
\item
  Vertebral artery takes origin from the subclavian artery in a very
  proximal position or is dominant over the contralateral side, then
  bypass preferred. \citep{moraschTechniqueSubclavianCarotid2009d}
\item
  For coronary-subclavian steal with patent internal mammary artery to
  coronary artery bypass graft, then Bypass (a carotid-subclavian
  transposition requires a more proximal clamp with occlusion of
  inline antegrade flow to the coronary bypass during the procedure)
  \citep{cuaReviewCoronarySubclavian2017}
\end{enumerate}

\hypertarget{vaso-occlusive-disease}{%
\section{Vaso-occlusive disease}\label{vaso-occlusive-disease}}

\textbf{What are causes and symptoms associated with subclavian/axillary
occlusive disease?} \citep{jacklcronenwettVascularDecisionMaking2020}

\begin{itemize}
\item
  Etiology: Atherosclerosis is the most common cause of
  subclavian/axillary occlusive disease. Left SCA \textgreater{} Right involvement.
  Less common causes include Takayasu disease, giant cell arteritis,
  or arterial TOS
\item
  Symptoms: Upper extremity arm/hand ischemia or neurologic symptoms
  due to subclavian-vertebral steal. Because significant collaterals,
  minimal pain on exertion even with subclavian occlusion
\end{itemize}

\textbf{What are causes and symptoms associated with brachial/forearm
occlusive disease?}

\begin{itemize}
\item
  Etiology: MCC of brachial artery occlusion is cardiac origin
  embolus. Atherosclerosis RARELY affects the brachial artery. Distal
  axillary/proximal brachial stenosis can be from repetitive trauma
  from crutch use.~
\item
  Forearm occlusive disease can be seen in advanced ESRD/DM where
  calcific atherosclerosis of radial/ulnar arteries is present. Less
  common causes include Beurger disease or Raynaud Phenomenon~
\end{itemize}

\textbf{How/when is upper extremity occlusive disease treated?}

\begin{itemize}
\item
  SCA Occlusive Disease

  \begin{itemize}
  \item
    Endovascular with balloon expandable stent via femoral or
    ipsilateral brachial artery.
    \[@chatterjeeAngioplastyAloneAngioplasty2013;
    @bradaricEndovascularTherapyStenoOcclusive2015;
    @sahaSubclavianArteryDisease2017\] Preferred in:

    \begin{itemize}
    \item
      Short segment or ostial disease with adequate distance to
      the vertebral artery origin.
    \item
      History of neck surgery or radiation.
    \end{itemize}
  \item
    Surgery:

    \begin{itemize}
    \item
      Bypass from aortic arch through median sternotomy~
    \item
      Ipsilateral CCA to subclavian artery (bypass or
      transposition)~
    \item
      Contralateral CCA (anterior or retropharyngeal)
    \end{itemize}
  \end{itemize}
\item
  Brachial/forearm Occlusive disease~

  \begin{itemize}
  \item
    Endovascular: PTA evidence is anecdotal with stents for lesions
    unresponsive to PTA or dissection following angioplasty~
  \item
    Surgery:~

    \begin{itemize}
    \tightlist
    \item
      GSV vein bypass remains standard for revascularization with
      bypasses to superficial or deep palmar arch have good
      patency rates. Tunneling is subcutaneous if to distal ulnar
      or superficial palmar arch whereas anatomical to distal
      radial artery over the anatomic snuffbox~
    \end{itemize}
  \end{itemize}
\end{itemize}

\hypertarget{vasospastic-disorders}{%
\section{Vasospastic Disorders}\label{vasospastic-disorders}}

\textbf{What is Raynaud's and what causes it?} \citep{shuja117UpperExtremity, landry141RaynaudPhenomenon2019}

\begin{itemize}
\tightlist
\item
  Exaggeration of normal physiologic response with episodic pallor or
  cyanosis of the fingers caused by small digital artery
  vasoconstriction occurring in response to cold or emotional stress.
  There is an abnormality with sympathetic nervous system, resulting
  in a multifactorial problem involving a combination of vascular,
  neural, and humoral factors.
\end{itemize}

\textbf{What are the subtypes of Raynaud's phenomenon and what is the
underlying pathology?}

\begin{itemize}
\item
  Primary: Raynaud's disease-idiopathic form that is a benign process
  not associated with structural vascular change. Triggers include
  (cold, emotional stress, caffeine) resulting in digital smooth
  muscle contraction and temporary digital hypoperfusion.
\item
  Secondary: Fixed vascular obstruction to blood flow decreasing
  threshold for cold induced vasospasm or progress to tissue loss.
  Diseases associated include mixed connective tissue disease, SLE,
  and rheumatoid arthritis, and scleroderma (accounts for 80-90\% of
  cases). In setting of lower digital blood pressure, symptomatic
  digital ischemia or tissue loss under low stress conditions. With
  cold/emotional stress, vasoconstrictive response of digital artery
  smooth muscle further causes arterial closure and resultant symptoms
\end{itemize}

\textbf{What are diagnostic criteria for Raynaud's?}

\begin{itemize}
\item
  Clinical (Progression of ischemia with white -\textgreater{} blue -\textgreater{} red finger
  discoloration. Episodes can be self-limited and may last from less
  than a minute, but generally not longer than 10-20 minutes~
\item
  Qualitative testing for severity of cold sensitivity in Raynaud's
  syndrome can be useful. Most basic test is cold sensitivity and
  recovery after ice water immersion. \textgreater10 minutes return to baseline
  pressure concerning for Raynaud's
\item
  Segmental pressures with finger systolic blood pressure can
  differentiate purely vasospastic vs occlusive disease. Difference of
  more than 15 mm Hg between fingers or absolute finger pressure \textless70
  mm Hg may indicate occlusive disease~
\item
  Serologic evaluation (ANA/RF)
\end{itemize}

\textbf{What are appropriate treatments for Raynaud's phenomenon?}

\begin{enumerate}
\def\labelenumi{\arabic{enumi}.}
\item
  Medical-cold/tobacco avoidance. Calcium channel blocker (nifedipine)
  has been the most effective and losartan has also been beneficial.
  Fluoxetine (SSRI). Other drugs include alpha blocker, sildenafil,
  reserpine, cilostazol, captopril. NOT GOOD OUTCOMES IN PATIENTS WITH
  ARTERIAL OBSTRUCTION~
\item
  Surgical-thoracic sympathectomy (used for treatment of digital
  artery vasospasm/digital ischemic ulceration). For vasospasm,
  thoracic sympathetcomy is initially successful, but symptoms return
  generally within 3-6 months.~
\item
  Immunosuppression/immunomodulation for connective tissue disorders
  associated with secondary Raynaud phenomenon
\end{enumerate}

\hypertarget{ergotism}{%
\subsection{\texorpdfstring{\textbf{Ergotism}}{Ergotism}}\label{ergotism}}

\textbf{What is Ergotism?} \citep{jamescstanleyCurrentTherapyVascular2014}

\begin{itemize}
\item
  Etiology: Ergot is a parasitic fungal disease that has a particular
  prevalence for infecting rye plants and ergot alkaloids have been
  linked to epidemic poisonings that manifested as ergotism from
  consumption of rye~
\item
  Modern day is rare
\end{itemize}

\textbf{What causes Ergotism and how do patients present?}

\begin{itemize}
\item
  Ergotamine is chemically like endogenous catecholamines/indolamines
  and when applied clinically, it behaves as an agonist to
  alpha-adrenergic, sertoninergic, and dopaminergic receptors. Despite
  limited bioavailability, vasocontrictive effects have been reported
  to last for 24 hours or longer~
\item
  Gangrenous-mild limb pain followed by burning pain/shooting and~
\item
  Convulsive-heaviness in limbs and head associated with diarrhea.
  Could result in tonic-clonic spasms~
\end{itemize}

\textbf{How can you diagnose Ergotism and what is the process for treating
this disease?}

Upper extremity ischemia (i.e.~digital ulceration) in the setting of
ergot alkaloid use (typically for migraines)~

Treatment:~

\begin{itemize}
\item
  Volume expansion and IV heparin as anticoagulation~
\item
  IV infusion of nitroprusside, nitroglycerin, ilioprost or
  combination
\item
  Infusion of Ca 2+ channel blockers~
\item
  Surgical: for thrombosis, consider thrombolysis~
\end{itemize}

\hypertarget{buergers-disease}{%
\subsection{\texorpdfstring{\textbf{Buerger's Disease}}{Buerger's Disease}}\label{buergers-disease}}

\textbf{How is Buerger's disease categorized?}
\citep{jacklcronenwettVascularDecisionMaking2020}

\begin{itemize}
\tightlist
\item
  Non-atherosclerotic, segmental, inflammatory disease of small/medium
  sized arteries in distal extremities of tobacco users distinct from
  either atherosclerosis of immune arteritis
\end{itemize}

\textbf{What clinical criteria can help diagnose Buerger's?}

\begin{itemize}
\tightlist
\item
  Smoking history, onset before 50 years, infrapopliteal arterial
  occlusions, upper limb involvement, absence of atherosclerotic risk
  factors besides smoking
\end{itemize}

\textbf{What is important about diagnosing Buerger's}

\begin{itemize}
\item
  Typically a diagnosis of exclusion
\item
  Must rule out proximal embolic source, trauma, local lesions (eg pop
  entrapment or cystic adventitial disease), autoimmune disease,
  hypercoagulable status, atherosclerosis
\end{itemize}

\textbf{What physical exam and non-invasive/invasive imaging findings of
Buerger's?}

\begin{itemize}
\item
  Distal, but not proximal arterial disease (palpable
  brachial/popliteal but absent/reduced at ankle or wrist)
\item
  DBI\textless0.6 and flat/reduced digital waveforms
\item
  CTA/MRA/DSA-characteristic corkscrew collateral
\end{itemize}

\textbf{What is the mainstay treatment in Buerger's disease?}

\begin{enumerate}
\def\labelenumi{\arabic{enumi}.}
\item
  Smoking cessation! Only treatment to improve symptoms and reduce
  amputation risk if achieved before onset of gangrene or tissue loss.
  Important to remember following treatments will likely fail without
  smoking cessation.~
\item
  If smoking cessation does not improve, medical management with
  antiplatelet agents, immunomodulators, vasodilators, anticoagulants~
\item
  Endovascular-distal small vessel intervention
\item
  Surgical-upper extremity autogenous vein bypass-limited success due
  to poor outflow~
\item
  Sometimes can consider upper extremity sympathectomy, but unproven
  benefit~
\item
  Amputation-reported in 30-40\% who are followed longer than 5 years~
\end{enumerate}

\hypertarget{large-artery-vasculitis}{%
\subsection{\texorpdfstring{\textbf{Large Artery Vasculitis}}{Large Artery Vasculitis}}\label{large-artery-vasculitis}}

\textbf{What are common characteristics for patients who are suspected to have
a large vessel vasculitis?} \citep{shanmugam137VasculitisOther2019}

\begin{itemize}
\item
  Affect aorta and major branches~
\item
  Present with non-specific heterogenous symptoms making the diagnosis
  challenging. Most commonly, they present with systemic or
  constitutional symptoms (fatigue, fever, weight loss, arthralgias)
\item
  Frequently, diagnosis made with presence of constitutional symptoms,
  elevated inflammatory markers, and dedicated imaging (MRA, CTA, DUS,
  or PET)
\end{itemize}

\textbf{How can you differentiate takayasu arteritis vs giant cell
arteritis?}

\begin{enumerate}
\def\labelenumi{\arabic{enumi}.}
\item
  Takayasu arteritis~

  \begin{enumerate}
  \def\labelenumii{\arabic{enumii}.}
  \item
    Aorta and primary~
  \item
    Young patients \textless20 years and female in 80-90\% of cases, Asian
    populations
  \item
    Criteria (ACR)

    \begin{enumerate}
    \def\labelenumiii{\arabic{enumiii}.}
    \item
      Onset \textless40 years
    \item
      Claudication of an extremity~
    \item
      Decreased brachial pulse~
    \item
      \textgreater10 mmHg SBP between arms
    \item
      Bruit over subclavian arteries or aorta
    \item
      Arteriographic evidence of narrowing/occlusion in
      aorta/primary branches/or large upper/lower extremity
      arteries
    \end{enumerate}
  \end{enumerate}
\item
  Giant cell arteritis~

  \begin{enumerate}
  \def\labelenumii{\arabic{enumii}.}
  \item
    Aorta and main branches, but pre-dilection for carotid artery
    branches
  \item
    Diagnosis:~

    \begin{enumerate}
    \def\labelenumiii{\arabic{enumiii}.}
    \item
      Age at disease onset \textgreater{} 50 years~
    \item
      New headache
    \item
      Temporal artery abnormality~
    \item
      Elevated ESR (\textgreater50)~
    \item
      Abnormal artery biopsy (gold standard test)
    \end{enumerate}
  \item
    Other symptoms include jaw pain with mastication or visual
    changes
  \item
    Associated with Polymyalgia rheumatic, characterized by morning
    stiffness in shoulders/hips occurring in 40-50\% of patients~
  \item
    Arteriography/MRA/CTA/PET may be used to assess large vessel
    involvement
  \end{enumerate}
\end{enumerate}

\textbf{How should patients be monitored with active large artery
vasculitis?}

\begin{itemize}
\item
  Lab data tracked at least monthly for 6 months with close follow-up
  to ensure appropriate response to medical treatment and enable
  physicians to assess for adverse effects of medical treatment
\item
  Repeat tests after remission reached and imaging choice to evaluate
  large vessels (DUS/CTA/MRA)
\end{itemize}

\textbf{What is the medical treatment for GCA and when do you consider
surgical treatment?}

\begin{itemize}
\item
  Medical-steroid therapy. In as many as 50\% of patients who have a
  large vessel vasculitis refractory to glucocorticoid therapy alone,
  patients will trial immunomodulators or cytotoxic truxs (ie
  methotrexate, azathioprine, mycophenolate, tocilizumab, or
  leflunomide)~
\item
  Intervention-once remission, treatment of symptomatic arterial
  lesions should be considered and as many as 50-70\% with large vessel
  vasculitis will require intervention.~

  \begin{itemize}
  \item
    Endovascular-angioplasty/stent/stent graft for large vessel
    vasculitis have all been described, however higher restenosis in
    endovascular compared to open treatment
  \item
    Open Surgery (gold standard)-lesions are long, fibrotic and
    therefore less amenable to endovascular treatment. Bypass grafts
    from aorta-CCA are the most common (CEA should be avoid due to
    pathology involved)

    \begin{itemize}
    \item
      Upper extremity bypass with autogenous vein to the brachial
      artery
    \item
      Aortic aneurysms should be managed with open surgery~
    \end{itemize}
  \end{itemize}
\end{itemize}

\hypertarget{aneurysmal-disease}{%
\section{Aneurysmal Disease}\label{aneurysmal-disease}}

\textbf{How are subclavian aneurysms caused and how can they present?}
\citep{baig84UpperExtremity2019}

Etiology/Pathology:

\begin{itemize}
\item
  Degenerative (atherosclerotic or due to aberrant right subclavian
  with degenerative changes in proximal subclavian known as ``Kommerell
  diverticulum'')
\item
  Traumatic (blunt, penetrating, iatrogenic with attempted catheter
  placement)~
\item
  Thoracic outlet obstruction
\end{itemize}

Presentation

\begin{itemize}
\tightlist
\item
  Exam-pulsatile supraclavicular mass or bruit, absent/diminished
  pulses, signs of microembolization (``blue finger'')
\item
  Most discovered incidentally, however referred chest, neck, shoulder
  pain, upper extremity ischemia due to thromboembolic phenomenon,
  brachial plexus compression, hoarseness from right recurrent
  laryngeal nerve compression
\item
  Dysphagia from esophageal compression in aberrant right subclavian
  artery
\end{itemize}

\textbf{What are diagnostic studies and treatment modalities for subclavian
aneurysms?}

\begin{itemize}
\item
  CXR-mediastinal mass may suggest neoplasm
\item
  MRA/CTA important to delineate extent of aneurysm and proximity to
  ipsilateral vertebral artery~
\end{itemize}

Treatment:~

\begin{itemize}
\item
  Open Repair-resection/endoaneurysmorrhaphy with end to end (small
  aneurysms) or interposition prosthetic graft~

  \begin{itemize}
  \item
    Proximal-median sternotomy with supraclavicular fossa extension
    for adequate proximal control for right side, however
    supraclavicular with left anterolateral thoracotomy for left
    subclavian aneurysm~
  \item
    Mid-Distal-supraclavicular/infraclavicular generally adequate
    for control where again resection of the clavicle may be needed~
  \end{itemize}
\item
  Endovascular Repair-transbrachial/transfemoral approach with covered
  stent~

  \begin{itemize}
  \tightlist
  \item
    Must consider vertebral artery origin. Can cover vertebral
    artery if contralateral vertebral artery is patent and of
    adequate size, however posterior circulation stroke may occur
    when the contralateral vertebral artery is highly stenotic,
    hypoplastic or occluded.
  \end{itemize}
\item
  Hybrid Repair-embolization/coils of proximal subclavian artery
  combined with subclavian transposition or carotid-subclavian bypass
\item
  For aberrant subclavian artery aneurysm, resection or exclusion of
  the aneurysmal artery with vascular reconstruction of the subclavian
  artery is recommended. Especially in the setting of dysphagia
  lusoria, subclavian artery reconstructed by interposition graft
  where proximal anastomosis is on ascending aorta. Alternatively,
  left posterolateral thoracotomy for proximal aneurysm resection and
  right supraclavicular incision for reconstruction of subclavian
  artery by end to side to the right CCA has been reported.
\end{itemize}

\textbf{How are axillary aneurysms caused and how can they present?}

Etiology/Pathology:~

\begin{itemize}
\item
  Blunt/penetrating trauma
\item
  Congenital (infrequently reported)
\item
  Post-traumatic axillary aneurysms (repeated abduction/external
  rotation downward toward humeral head in baseball pitchers)~
\end{itemize}

Presentation:~

\begin{itemize}
\tightlist
\item
  Exam-pulsatile supraclavicular mass or bruit, absent/diminished
  pulses, signs of microembolization (``blue finger'')
\end{itemize}

\textbf{What are diagnostic studies and treatment modalities for axillary
aneurysms?}

Diagnosis:~

\begin{itemize}
\item
  Ultrasound
\item
  CTA/MRA of upper extremity
\end{itemize}

Treatment:~

\begin{itemize}
\item
  Open Repair-resection with interposition vein grafting or prosthetic
  if inadequate vein is present.~
\item
  Endovascular repair-covered stent graft can be placed with
  occasional embolization with micro coils to isolate sac and prevent
  retrograde endoleaks
\end{itemize}

\textbf{How are brachial aneurysms caused and how can they present?}

Etiology/Pathology:~

\begin{itemize}
\item
  False aneurysms secondary to repetitive trauma
\item
  Iatrogenic complications~
\item
  IV drug abuse (infected pseudoaneurysms in antecubital fossa)~
\item
  Connective tissue disorders (ex. type IV Ehlers danlos)
\end{itemize}

Presentation:~

\begin{itemize}
\item
  Exam: pulsatile mass~
\item
  Local pain or symptoms of median nerve compressions
\item
  Hand/digital ischemia from thrombosis/distal embolization
\end{itemize}

\textbf{What are diagnostic studies and treatment modalities for brachial
aneurysms?}

Diagnosis:~

\begin{itemize}
\item
  Duplex Ultrasound
\item
  CTA/MRA of upper extremity may be needed to delineate extent of
  aneurysm~
\end{itemize}

Treatment:~

\begin{itemize}
\item
  Open Repair (preferred)-resection with patch or interposition vein
  grafting~~
\item
  Endovascular repair-rare and generally in a traumatic setting
\item
  Iatrogenic injuries-due to access and nonoperative treatment for
  small/asymptomatic pseudoaneurysms that are likely to thrombose
  spontaneously. Direct suture repair with evacuation of hematoma is
  possible. Thrombin injection is less favorable due to location and
  short neck.~
\end{itemize}

\hypertarget{occupational-vascular-disease}{%
\section{Occupational Vascular Disease}\label{occupational-vascular-disease}}

\textbf{There are some occupational vascular disorders than contribute to
vascular disease in the upper extremity. Hand arm vibration syndrome and
hypothenar hammer are of particular importance. Can you talk to us about
the key information from these syndromes?}
\citep{eskandari185ConditionsArising2020}

\hypertarget{hand-arm-vibration-syndrome}{%
\subsection{Hand-Arm Vibration Syndrome~}\label{hand-arm-vibration-syndrome}}

Etiology:

\begin{itemize}
\item
  Vibrating handheld machines (eg pneumatic hammers and drills,
  grinders, and chain saws)~
\item
  Linear relationship between exposure over years and onset of this
  syndrome~
\item
  Exact mechanism unknown, but thought that endothelial damage with
  sympathetic hyperactivity -\textgreater{} finger blanching attack~
\end{itemize}

Presentation:

\begin{itemize}
\item
  Various stages seen where early results in slight tingling/numbness
  and lateral, the tips of one or more fingers experience attacks of
  blanching that is usually precipitated by cold~
\item
  Blanching typically lasts 1 hour and terminates with reactive
  hyperemia, but prolonged exposure can cause bluish black cyanosis of
  fingers~
\end{itemize}

Diagnosis~

\begin{itemize}
\item
  Detailed history with use of vibrating tools/symptoms of Raynaud
  phenomenon~
\item
  Objectively: cold induced ischemia with recording time until digital
  temperature recovers
\item
  Digital occlusion with noninvasive digit pressures or duplex
  scanning
\end{itemize}

Treatment

\begin{itemize}
\item
  Avoidance of vibratory tools
\item
  Nifedipine (Ca2+ channel blocker) in advanced cases~
\item
  IV prostanoid (ie prostacyclin) for digital gangrene~
\item
  Surgery-cervical sympathetectomy or digital sympathectomy rarely
  needed\\
\end{itemize}

\hypertarget{hypothenar-hammer-syndrome}{%
\subsection{Hypothenar hammer syndrome}\label{hypothenar-hammer-syndrome}}

Etiology:~

\begin{itemize}
\item
  Repetitive use of palm of hand in activities that involve pushing,
  pounding, twisting
\item
  Name comes from reports of mechanics, factory workers, carpenters or
  laborers who habitually use there hands as a hammer are ad risk for
  disease~
\item
  Repetitive trauma leads to thrombotic occlusion, aneurysm formation
  or both
\end{itemize}

Presentation:~

\begin{itemize}
\item
  Asymmetrical distribution involving dominant upper extremity where
  cyanosis and pallor can occur and digits affected are ulnar
  distribution in nature~
\item
  Cool/mottled digits or severe cases with ischemic ulcers
\end{itemize}

Diagnosis:~

\begin{itemize}
\item
  Duplex ultrasound~
\item
  CTA or MRA~
\item
  Arteriography (gold standard) with corkscrew pattern typically in
  affected vessels
\end{itemize}

Treatment~

\begin{itemize}
\item
  Conservative-smoking cessation/hand protection/cold avoidance~
\item
  Medical-calcium channel blockers/antiplatelet
\item
  Surgical (severe digital ischemia/aneurysm)-ligation if adequate
  collateral or interposition vein graft
\end{itemize}

\hypertarget{environmental-exposures}{%
\subsection{Environmental Exposures}\label{environmental-exposures}}

\textbf{Exposure to what environmental agents can result in upper extremity
ischemia?}

Acrosteolysis

\begin{itemize}
\item
  Exposure to polyvinyl chloride can result in ischemic hand symptoms
  similar to those of Raynaud syndrome~
\item
  Angiography-damage to digital arteries with multiple
  stenosis/occlusions or hyper vascularity adjacent to areas of bone
  resorption~
\item
  Treatment-supportive
\end{itemize}

Electrical burns

\begin{itemize}
\item
  \textless1000 V cause injuries limited to immediate skin/soft tissue,
  however \textgreater1000 V cause damage from entry to exit point~
\item
  Results in arterial necrosis with thrombus or bleeding and gangrene
  of digits develop~
\item
  Initially can be occlusion/thrombosis or spasm, however later damage
  can cause aneurysmal degeneration~
\item
  Treatment-dependent on soft tissue/bone injuries as well. Can have
  reconstruction with free flap due to local vascular damage or
  occlusion of major artery requiring bypass grafting
\end{itemize}

Extreme thermal injuries

\begin{itemize}
\item
  Workers at risk with chronic exposure to cold (slaughterhouse,
  canning factory, and fisheries)
\item
  Raynaud syndrome symptoms due to vasomotor disturbances in the hands
  when exposed to extreme chronic thermal trauma
\item
  Treatment-Supportive~
\end{itemize}

\hypertarget{sports-medicine}{%
\subsection{Sports Medicine}\label{sports-medicine}}

\textbf{How can athletes specifically be affected by upper extremity
ischemia?}

Overview

\begin{itemize}
\tightlist
\item
  Athletes who engage in strenuous or exaggerated hand/shoulder
  activity may be susceptible to upper extremity ischemia from
  arterial injury manifested by Raynaud syndrome, symptoms of sudden
  arterial occlusion or digital embolization
\end{itemize}

\hypertarget{vascular-trauma-upper-extremity}{%
\section{Vascular Trauma-Upper Extremity}\label{vascular-trauma-upper-extremity}}

\textbf{This is discussed in detail here:} \ref{vascular-trauma}, \textbf{so we
will go over some important specifics for upper extremity vascular
injury.} \citep{kauvar184VascularTrauma2020}

\textbf{What is the mechanism and management of upper extremity axillary
artery trauma?}

Mechanism and Pattern

\begin{itemize}
\tightlist
\item
  Predominantly in penetrating trauma with equal incidence in
  proximal/middle/distal divisions and brachial plexus injury
  in \textgreater1/3rd of arterial injury
\end{itemize}

Diagnostic Considerations

\begin{itemize}
\item
  Physical exam with deficiencies in upper extremity pulses/ischemic
  changes, but may not be present given collateral flow from axillary
  artery to upper extremity~
\item
  High index of suspicion with location of injury proximity to course
  of axillary artery
\item
  Upper extremity Doppler or CTA if patient is stable for diagnosis
\end{itemize}

Surgical Considerations~

\begin{itemize}
\item
  Primary repair or treated with interposition graft~
\item
  If hemodynamically stable, can consider covered stent based on
  location to thoracic outlet via femoral/brachial approach
\end{itemize}

\textbf{What is the mechanism and management of upper extremity brachial
artery trauma?}

Mechanism and Pattern

\begin{itemize}
\item
  Frequently associated with humerus fractures/elbow dislocation
\item
  Penetrating trauma
\end{itemize}

Diagnostic Considerations

\begin{itemize}
\item
  Pulse deficit in majority (\textgreater75\% of cases)
\item
  Upper extremity Doppler of CTA
\end{itemize}

Surgical Considerations~

\begin{itemize}
\tightlist
\item
  Given course, can be extensively mobilized and repaired in
  end-to-end fashion in 50\% of cases. Otherwise, treatment with an
  interposition graft
\end{itemize}

\textbf{What is the mechanism and management of upper extremity radial/ulnar
artery trauma?}

Mechanism and Pattern

\begin{itemize}
\tightlist
\item
  Associated with significant soft tissue pattern
\end{itemize}

Diagnostic Considerations

\begin{itemize}
\item
  Pulse deficit in \textgreater80\% of patients~
\item
  Doppler based Allen test-confirm radial/ulnar contribution to palmar
  arch
\end{itemize}

Surgical Considerations~

\begin{itemize}
\item
  If Allen test reveals a patent palmar arch, the injured artery can
  be ligated~
\item
  If palmar arch is not patent in the absence of contribution of the
  injured artery, it should be repaired
\item
  If both are damaged, preference to ulnar artery as dominant
  contribution to hand~
\item
  Generally, repair can be done in an end to end fashion given
  mobility of the vessel
\end{itemize}

\hypertarget{compression-syndromes}{%
\section{Compression Syndromes}\label{compression-syndromes}}

\textbf{The} \textbf{main syndromes are quadrilateral space syndrome and humeral
compression of the axillary artery. What important information here do
our listeners need to know?}

\hypertarget{quadrilateral-space-syndrome}{%
\subsection{Quadrilateral space syndrome}\label{quadrilateral-space-syndrome}}

Anatomy:

\begin{itemize}
\item
  Bordered by teres minor superiorly, humeral shaft laterally, and
  teres major inferiorly, and long head of triceps muscle medially
\item
  Posterior humeral circumflex artery and axillary nerve in space
\end{itemize}

Pathophysiology

\begin{itemize}
\item
  Compression of posterior humeral circumflex occurs with
  abduction/external rotation~
\item
  Typically seen with chronic overhand motion athletes
  (pitchers/volleyball players)
\item
  Vascular-repetitive mechanical trauma to posterior circumflex
  humeral artery~
\item
  Neurogenic-fixed structural impaction of quadrilateral space by
  fibrous bands or space-occupying lesions
\end{itemize}

Presentation

\begin{itemize}
\tightlist
\item
  Muscle atrophy, paresthesias, poorly localized shoulder pain and
  pain in quadrilateral space
\end{itemize}

Treatment

\begin{itemize}
\item
  Medical: Oral anti-inflammatory medications, PT, limitation of
  activities
\item
  Surgery: decompression with neurolysis/excision of fibrous bands or
  other space occupying lesions~
\end{itemize}

\hypertarget{humeral-head-compression-of-axillary-artery}{%
\subsection{Humeral head compression of axillary artery}\label{humeral-head-compression-of-axillary-artery}}

Anatomy:

\begin{itemize}
\tightlist
\item
  3rd portion of axillary artery compressed by head of humerus
\end{itemize}

Etiology/Pathophysiology:

\begin{itemize}
\tightlist
\item
  Arm is abducted and externally rotated with downward compression of
  humeral head to axillary artery
\end{itemize}

Presentation:

\begin{itemize}
\tightlist
\item
  Arm fatigue, loss of pitch velocity, finger numbness, Raynaud,
  cutaneous embolization
\end{itemize}

Diagnosis:

\begin{itemize}
\item
  Provocative maneuvers with impedance of flow through axillary artery
  on ultrasonography
\item
  Arteriography with rest and provocative position
\end{itemize}

Treatment:

\begin{itemize}
\item
  Supportive with avoidance of throwing motion
\item
  Surgical-saphenous vein patch for no improvement or structural
  injury may require resection with saphenous vein bypass anatomically
  or extra-anatomic tunneling above pec minor
\end{itemize}

\hypertarget{thoracic-outlet-syndrome}{%
\section{Thoracic Outlet Syndrome}\label{thoracic-outlet-syndrome}}

\textbf{27 Nov 2019:} \emph{Nedal Katib, Prince of Wales, Sydney Australia}

Thoracic Outlet Syndrome = A constellation of signs and symptoms
relating to the compression of the neurovascular structures that occurs
as these structures travel between the thoracic aperture and the upper
limb.

Types: Neurogenic, Venous and Arterial~

\begin{itemize}
\item
  vTOS -- 2-3\%
\item
  aTOS -- 1\%
\item
  nTOS --~ \textgreater95\% \citep{humphries124ThoracicOutlet2019}
\end{itemize}

\hypertarget{anatomy}{%
\subsection{Anatomy}\label{anatomy}}

Understanding the anatomy of what is collectively referred to as the
thoracic outlet is the best way to thoroughly appreciate this topic.

Anatomy from anterior to posterior

\begin{itemize}
\item
  Subclavian vein
\item
  Phrenic nerve
\item
  Anterior scalene muscle attachment to the first rib
\item
  The subclavian artery
\item
  The brachial plexus
\item
  The middle scalene muscle.
\end{itemize}

Three spaces where the neurovascular structures are at risk of
compression:

\begin{enumerate}
\def\labelenumi{\arabic{enumi}.}
\item
  Interscalene Triangle~
\item
  Costoclavicular Passage~\citep{garygwindAnatomicExposuresVascular2013}
\item
  Subcoracoid Space \citep{garygwindAnatomicExposuresVascular2013}
\end{enumerate}

\textbf{Interscalene Triangle:}

Appreciating the attachments of the Anterior and Middle Scalene Muscles
on the first rib becomes important in the diagnosis of the various types
and also the ultimate surgical management of the compression.

\textbf{Anterior Scalene:}

Attachments: Anterior Tubercles of the four `typical' cervical vertebrae
(3-6) AND the scalene tubercle on the upper surface of the first rib.

\begin{itemize}
\tightlist
\item
  Phrenic nerve runs along anterior scalene muscle and injury can
  cause ipsilateral diaphragm paralysis.
\end{itemize}

\textbf{Middle Scalene:}

Attachments: The posterior tubercles and intertubercular lamellae of all
the cervical vertebrae AND the Quadrangular area between the neck and
subclavian groove of the first rib. \citep{mcminnLastAnatomyRegional2019}

\begin{itemize}
\tightlist
\item
  Long thoracic nerve runs along middle scalene muscle and injury can
  cause winged scapula.
\end{itemize}

\textbf{The First Rib:}~

\begin{itemize}
\item
  The broadest and flattest of the ribs and is an `Atypical Rib'.~
\item
  The upper surface of the first rib has the scalene and quadrangular
  tubercles for attachments of the anterior and middle scalene muscles
  respectively. There are also three grooves for the Subclavian Vein,
  artery and the Lower Trunk of the Brachial Plexus.~
\item
  The Inferior Surface is smooth and inferior and medially has an
  attachment for the suprapleural membrane, Sibson's fascia AKA
  scalenus minimus, which is tethered to the C7 vertebrae.~
\item
  This is the passage of the subclavian vein largely as it emerges
  through the tight space created by the clavicle, the subclavius
  muscle and the costoclavicular ligament and also more posteriorly
  this can also compress the artery and nerves as the space can also
  be narrow in relation with the scapula and subscapularis.
  \citep{garygwindAnatomicExposuresVascular2013}
\end{itemize}

\textbf{Subclavius Muscle:}

\begin{itemize}
\item
  Attached to the costochondral junction of the first rib and is
  inserted into the subclavian groove on the inferior surface of the
  clavicle. \citep{mcminnLastAnatomyRegional2019}
\item
  This space is best appreciated by intimate knowledge of three
  things:

  \begin{itemize}
  \item
    The Coracoid Process and its attachments
  \item
    The Pectoralis Minor Muscle
  \item
    The Clavipectoral Fascia
  \end{itemize}
\end{itemize}

\textbf{The Coracoid Process:}~

\begin{itemize}
\item
  Arising from the Scapula as a `process', this broad-based bony
  landmark offers attachment to muscles and ligaments.
\item
  The relevant attachments being the pectoralis minor muscle occupying
  the medial border for about 2cm behind its tip. The tip itself
  having a medial and lateral facet for the short head of biceps and
  the coracobrachialis muscles respectively.
\end{itemize}

\textbf{Pectoralis Minor Muscle:}

Attached to the bone of the third, fourth and fifth ribs AND the medial
border of the coracoid process.~

\textbf{Clavipectoral Fascia:}

A sheet of fascial membrane that fills the space between the clavicle
and pectoralis minor splitting and encompassing the subclavius muscle.
Its superior portion is what can be thickened and become a tight band
referred to as the costocoracoid ligament.~

\textbf{Phrenic Nerve Anomaly:}

The Phrenic Nerve normally runs anterior to the Subclavian Vein. A rare
anomaly is the nerve compressing the vein anteriorly and in very rare
circumstances due to the timing of development can run through the vein
itself.

Anomalous anatomy can also cause TOS especially when patients have a
Cervical Rib and anomalous first ribs or a congenital band attaching to
the first rib.

\begin{itemize}
\item
  Incidence of anomalous first ribs and cervical ribs is 0.76\% and
  0.75\% respectively.~
\item
  Incidence of bands are as high as 63\% in the general population.
  \citep{humphries124ThoracicOutlet2019}
\end{itemize}

nTOS~~

\begin{itemize}
\item
  Scalene Triangle compression -- most common cause of brachial plexus
  and neurogenic TOS
\item
  Cervical Rib and Anomalous First Rib
\end{itemize}

aTOS

\begin{itemize}
\item
  Cervical Rib and Anomalous First Rib
\item
  Scalene Triangle compression
\end{itemize}

vTOS

\begin{itemize}
\item
  Costoclavicular Passage
\item
  Subcoracoid Space
\end{itemize}

\hypertarget{diagnosis-and-evaluation}{%
\subsection{Diagnosis and Evaluation}\label{diagnosis-and-evaluation}}

\hypertarget{patient-history}{%
\subsubsection{Patient History}\label{patient-history}}

\begin{itemize}
\item
  Identify symptoms and thoroughly interrogate timing
\item
  Exclude history of trauma
\item
  Associated symptoms like headache, visual disturbance, neurology in
  the upper limb
\item
  Exclude Carpal Tunnel and Antecubital Tunnel Syndromes if symptoms
  are isolated to the arm or forearm or hand
\item
  Patients with vTOS may present acutely and have acute or subacute
  Upper Limb DVT
\item
  Patients with aTOS need to be investigated and assessed urgently
  given risk of ischemia.~
\end{itemize}

\hypertarget{clinical-examination}{%
\subsubsection{Clinical Examination}\label{clinical-examination}}

Provocative maneuvers are largely used for nTOS. While these are
described and mentioned in most texts their utility largely is beyond
the scope of a vascular surgeon's assessment and diagnosis of nTOS.

\textbf{Adson Test}

\begin{itemize}
\item
  Extended abducted and externally rotated arm -- palpate radial pulse
\item
  Rotate and laterally flex the neck to the ipsilateral side while
  inhaling deeply.
\item
  A positive test results in reduction or complete obliteration of
  radial pulse
\end{itemize}

\textbf{Roos Test / EAST test}

\begin{itemize}
\item
  Patient seated and both arms abducted 90 degrees and externally
  rotated and elbows flexed at 90 degrees.~
\item
  Open and close hands for 3 minutes or until pain or paraesthesia
  sets in.
\end{itemize}

\textbf{Elveys Test}

\begin{itemize}
\item
  Abduct both arms to 90 degrees with elbows extended and dorsiflex
  both wrists.
\item
  If pain is elicited as wrists dorsiflexed then test is positive.
\item
  A further manoeuvre is then performed, laterally flex the head on
  each side, if pain is elicited on the contralateral side to which
  the head is flexed then test is positive.
  \citep{humphries124ThoracicOutlet2019}
\end{itemize}

\hypertarget{non-invasive-imaging-or-vascular-lab-studies}{%
\subsubsection{Non-invasive imaging or vascular lab studies}\label{non-invasive-imaging-or-vascular-lab-studies}}

\begin{itemize}
\item
  DBI
\item
  Arterial Duplex
\item
  Venous Duplex~
\item
  CT -- CTV commonly performed in acute upper limb DVT and suspicion
  of vTOS
\item
  CTA for the evaluation of aTOS and excluding other causes of
  embolisation
\item
  MRI -- for further evaluation of the anatomy and related
  neurovascular compression
\item
  Electromyography and Nerve Conduction Studies for nTOS
\end{itemize}

\textbf{Paget Schroetter Syndrome}

\begin{itemize}
\item
  First defined by Hughes in 1949 in reference to Sir James Paget who
  in a hundred years earlier defined acute arm swelling and pain as
  possibly related to vasospasm and then von Schroetter who in 1884
  attributed to the presentation to subclavian and axillary vein
  thrombosis. \citep{humphries123ThoracicOutlet2019}
\item
  Now vTOS and Paget Schroetter Syndrome are used synonymously.
\item
  Paget Schroetter Syndrome accounts for 10-20\% of all upper extremity
  deep vein thrombosis. \citep{sekharYearbookVascularEndovascular2018}
\end{itemize}

\hypertarget{rib-resection-approaches}{%
\subsection{Rib Resection approaches}\label{rib-resection-approaches}}

\begin{longtable}[]{@{}
  >{\raggedright\arraybackslash}p{(\columnwidth - 4\tabcolsep) * \real{0.17}}
  >{\raggedright\arraybackslash}p{(\columnwidth - 4\tabcolsep) * \real{0.38}}
  >{\raggedright\arraybackslash}p{(\columnwidth - 4\tabcolsep) * \real{0.42}}@{}}
\toprule
\begin{minipage}[b]{\linewidth}\raggedright
\end{minipage} & \begin{minipage}[b]{\linewidth}\raggedright
Advantages
\end{minipage} & \begin{minipage}[b]{\linewidth}\raggedright
Disadvantages
\end{minipage} \\
\midrule
\endhead
Tran
saxillary & Cosmetically more
appealing as it has a
limited hidden scar & \begin{minipage}[t]{\linewidth}\raggedright
\begin{itemize}
\item
  Difficult to visualize
  the anatomy, dependent
  on good assistance
\item
  Risk of injury to T1
  nerve root, phrenic
  nerve, long thoracic,
  brachial plexus ,
  subclavian vein and
  arterial with limited
  exposure to repair
\item
  Not able to approach
  cervical ribs, scalene
  triangle or patch vein.
\end{itemize}
\end{minipage} \\
Suprac
lavicular & \begin{minipage}[t]{\linewidth}\raggedright
\begin{itemize}
\item
  Good for scalene
  triangle access and
  debulking and
  cervical rib
  resection
\item
  Required for aTOS if
  arterial
  reconstruction
  necessary
\end{itemize}
\end{minipage} & \begin{minipage}[t]{\linewidth}\raggedright
\begin{itemize}
\item
  Unable to decompress
  venous compression or
  visualize vein
  adequately
\item
  Cosmetically less
  appealing
\end{itemize}
\end{minipage} \\
Infrac
lavicular
{[}@
siracuseI
nfraclavi
cularFirs
tRib2015{]} & \begin{minipage}[t]{\linewidth}\raggedright
\begin{itemize}
\item
  Good access for
  venous decompression
\item
  Allows for excision
  of subclavius muscle
  and costoclavicular
  ligament
\end{itemize}
\end{minipage} & \begin{minipage}[t]{\linewidth}\raggedright
\begin{itemize}
\item
  Unable to expose
  subclavian artery or
  decompress brachial
  plexus.
\item
  Difficult to access
  most posterior aspect
  of rib
\item
  Cosmetically less
  appealing
\end{itemize}
\end{minipage} \\
Parac
lavicular & \begin{minipage}[t]{\linewidth}\raggedright
\begin{itemize}
\tightlist
\item
  Useful if mixed
  etiology TOS to
  adequately
  decompression all
  neurovascular
  structures
\end{itemize}
\end{minipage} & \begin{minipage}[t]{\linewidth}\raggedright
\begin{itemize}
\tightlist
\item
  Requires two incisions
  one above and below the
  clavicle
\end{itemize}
\end{minipage} \\
\bottomrule
\end{longtable}

\hypertarget{post-operative-complications}{%
\subsubsection{Post operative complications}\label{post-operative-complications}}

\begin{itemize}
\item
  Post operative patients with hemodynamic instability and ipsilateral
  effusion on xray should go back to OR for exploration and hemorrhage
  control. \citet{rinehardtCurrentPracticeThoracic2017}
\item
  Chyle leak often managed with adequate drainage and medium chain
  fatty acid diet.
\end{itemize}

\hypertarget{vtos}{%
\subsection{vTOS}\label{vtos}}

\hypertarget{demographics}{%
\subsubsection{Demographics}\label{demographics}}

\begin{itemize}
\item
  Incidence: 2/100,000 persons
\item
  Age: 18 years to 30 years
  \citep{illigComprehensiveReviewPagetSchroetter2010}
\item
  M\textgreater F
\end{itemize}

\hypertarget{presentation}{%
\subsubsection{Presentation}\label{presentation}}

\begin{itemize}
\item
  Upper Limb edema, pain and cyanosis. Edema affects the shoulder, arm
  and hand and characteristically non pitting.
\item
  Collateral vein dilatation over the shoulder, neck and anterior
  chest wall to accommodate for the increased venous hypertension.
  \citep{humphries123ThoracicOutlet2019}
\item
  Pain on exertion of the upper limb described as stabbing, aching or
  tightness.
\item
  The reported incidence of PE following Upper Limb DVT is \textless12\%.
  \citep{humphries123ThoracicOutlet2019}
\end{itemize}

\hypertarget{history}{%
\subsubsection{History}\label{history}}

\begin{itemize}
\item
  A differential diagnosis for Upper Limb DVT

  \begin{itemize}
  \item
    vTOS
  \item
    Congenital Phrenic Nerve anomaly
  \item
    History of Fracture, Clavicular Fracture and malunion
  \item
    Repetitive arm provocative manoeuvres, check occupation and
    history of body-building

    \begin{itemize}
    \tightlist
    \item
      Pectoralis Minor Hypertrophy.
    \end{itemize}
  \end{itemize}
\item
  Exclude Pulmonary Embolism~
\item
  Exclude Venous Gangrene and Phlegmasia of the upper limb
\end{itemize}

\hypertarget{goals-of-therapy-for-vtos}{%
\subsubsection{Goals of therapy for vTOS}\label{goals-of-therapy-for-vtos}}

Limited evidence due to lack of RCT's. Majority of evidence based on
retrospective studies.~

\begin{itemize}
\item
  Prevent immediate risk~
\item
  Return patient to unrestricted use of the affected extremity
\item
  Prevent recurrence of thrombosis without the need of long-term
  anticoagulation
\item
  Prevent long term Post Phlebitic Limb Syndrome
\end{itemize}

\hypertarget{initial-management-strategy}{%
\subsubsection{Initial management strategy}\label{initial-management-strategy}}

\begin{itemize}
\item
  As per ACCP Guidelines: Initial management is anticoagulation
  regardless of etiology. \citep{kearonAntithromboticTherapyVTE2016}

  \begin{itemize}
  \item
    The limitations of anticoagulation alone are that the slow
    recanalization of the thrombus may lead to eventual valvular
    damage and intravenous scarring.
    \citep{sekharYearbookVascularEndovascular2018}
  \item
    Thrombolysis has been considered superior to anticoagulation
    alone in minimizing valvular damage due to residual clot.
    \citep{urschelSurgeryRemainsMost2008}
  \item
    Systemic Lysis -- non favored due to risk of intracranial
    hemorrhage. \citep{grunwaldCatheterDirectedThrombolysisTreatment2004}
  \item
    Catheter Directed Lysis (CDT) -- carries a lower risk of
    intracranial hemorrhage.
  \item
    Patient should be maintained in a compression sleeve until
    definitive decompression can be performed.
  \end{itemize}
\item
  Optimal timing of CDT

  \begin{itemize}
  \tightlist
  \item
    Within 14 days of onset of thrombosis. Excellent results have
    been reported following CDT if initiated before 14 days.
    \citep{wilsonFibrinolyticTherapyIdiopathic1990}
  \end{itemize}
\item
  Surgical indications for vTOS

  \begin{itemize}
  \item
    After initial management patients are generally divided into two
    groups, unsuccessful or successful thrombolysis.
  \item
    Persistent stenosis or signs of extrinsic compression, on
    venography, has generally been perceived as a significant risk
    of recurrent thrombosis.
  \item
    Surgery for vTOS remains to be mainly Rib Resection and
    decompression of the subclavian vein with or without venolysis
    and patch plasty either surgical or endovenous.
  \item
    Surgical treatment of severe resistant subclavian vein stenosis
    in the setting of vTOS is rib resection by paraclavicular
    approach and vein patch plasty.
    \[@melbyComprehensiveSurgicalManagement2008\]
  \item
    Venous occlusion in vTOS may be treated with jugular turn down
    or venous bypass to IJ of SVC if patients remain symptomatic.
    \citep{vemuriDiagnosisTreatmentEffortinduced2016}
  \end{itemize}
\end{itemize}

\hypertarget{controversy-around-vtos}{%
\subsubsection{Controversy around vTOS}\label{controversy-around-vtos}}

\begin{itemize}
\item
  There is a lack of consensus around the necessity of surgical rib
  resection, the timing and the requirement for vein patch plasty.
\item
  Options post recanalization:

  \begin{itemize}
  \item
    Deferring surgical decompression for 1-3 months after
    thrombolysis to allow for healing of the venous endothelium and
    resolution of the acute inflammatory process.
    \citep{humphries123ThoracicOutlet2019}
  \item
    Decompression during the same admission, as the thrombolysis,
    with the main benefit being to reduce the risk of re-occlusion.
    \citep{humphries123ThoracicOutlet2019, molinaPagetSchroetterSyndromeTreated2007}
  \item
    Post decompression venography and treatment 2 weeks post rib
    resection may help to prevent recurrence and long term vein
    patency. \citep{changRoutineVenographyFollowing2012}
  \end{itemize}
\end{itemize}

\hypertarget{landmark-papers-regarding-vtos-and-what-are-the-take-home-messages}{%
\subsubsection{Landmark papers regarding vTOS and what are the take home messages}\label{landmark-papers-regarding-vtos-and-what-are-the-take-home-messages}}

\begin{enumerate}
\def\labelenumi{\arabic{enumi}.}
\item
  Lugo J et al -- Acute Paget Schroetter syndrome: does the first rib
  routinely need to be removed after thrombolysis? Annals of Vascular
  Surgery 2015 \citep{lugoAcutePagetSchroetter2015}

  \begin{enumerate}
  \def\labelenumii{\arabic{enumii}.}
  \item
    Systematic literature review analysis. Patients divided into
    three groups

    \begin{enumerate}
    \def\labelenumiii{\arabic{enumiii}.}
    \item
      First Rib resection (FRR) -- n=448
    \item
      First Rib resection and endovenous venoplasty (FRR and
      PLASTY) n=68
    \item
      No further intervention after Thrombolysis n=168
    \end{enumerate}
  \item
    Symptom relief after initial follow up more likely in FRR (95\%)
    and FRR and PLASTY (93\%) compared to no rib removed (54\%) --
    p\textless0.0001
  \item
    Results showed superior patency with FRR and PLASTY and FRR
    compared to anticoagulation alone.~
  \item
    Conclusion was that patients are more likely to experience
    greater long-term results with FRR compared to no FRR.~
  \end{enumerate}
\item
  Sajid MS et al -- Upper limb vein thrombosis: a literature review to
  streamline the protocol for management. Acta Haematology 2007
  \citep{sajidUpperLimbDeep2007}

  \begin{enumerate}
  \def\labelenumii{\arabic{enumii}.}
  \tightlist
  \item
    a comprehensive review identifying the key papers on this topic
    and allows for a clear view of the best management strategy.
  \end{enumerate}
\item
  Vemuri, C., Salehi, P., Benarroch-Gampel, J., McLaughlin, L. N., \&
  Thompson, R. W. (2016). Diagnosis and treatment of effort-induced
  thrombosis of the axillary subclavian vein due to venous thoracic
  outlet syndrome. \emph{Journal of Vascular Surgery: Venous and Lymphatic
  Disorders}, \emph{4}(4), 485--500.
  \citep{vemuriDiagnosisTreatmentEffortinduced2016}

  \begin{enumerate}
  \def\labelenumii{\arabic{enumii}.}
  \tightlist
  \item
    Comprehensive summary of management strategy for effort induced
    thrombosis.
  \end{enumerate}
\end{enumerate}

\hypertarget{atos}{%
\subsection{aTOS}\label{atos}}

\hypertarget{presentation-1}{%
\subsubsection{Presentation}\label{presentation-1}}

\begin{itemize}
\item
  Most common: Hand ischemia due to arterial compression or
  microembolization with subclavian artery aneurysm and pulsatile
  supraclavicular mass \citep{boll122ThoracicOutlet2019}
\item
  Less common: Exertional pain, unilateral Raynaud's Phenomena,
  retrograde embolisation and neurological symptoms~
\item
  Clinical Examination

  \begin{itemize}
  \item
    Audible Bruit / Palpable thrill over the supraclavicular fossa
  \item
    Pulsatile mass
  \item
    Distal ischemic lesions in the distal hand -- Splinter
    hemorrhages
  \item
    Positive Adson Test
  \end{itemize}
\item
  Differential Diagnosis

  \begin{itemize}
  \item
    Trauma
  \item
    Primary and Secondary Raynaud's Phenomena
  \item
    Small Vessel Vasculitis
  \item
    Connective Tissue Disorders
  \item
    Thromboangiitis Obliterans
  \item
    Arterial Embolisation -- Aortic or Central Source
  \item
    Radiation Arteritis
  \item
    Atherosclerotic / Dissection causes
  \end{itemize}
\item
  The different anatomical abnormalities causing aTOS
  \citep{boll122ThoracicOutlet2019}

  \begin{itemize}
  \item
    Cervical Rib (60\%)
  \item
    Anomalous First Rib (18\%)
  \item
    Fibrocartilaginous band (15\%)
  \item
    Clavicular Fracture (6\%)
  \item
    Enlarged C7 transverse process (1\%)
  \end{itemize}
\end{itemize}

\hypertarget{scher-staging-of-atos}{%
\subsubsection{Scher Staging of aTOS}\label{scher-staging-of-atos}}

\begin{itemize}
\item
  Stage 0: Asymptomatic
\item
  Stage 1: Stenosis of Subclavian Artery with minor post stenotic
  dilatation with no intimal disruption
\item
  Stage 2: Subclavian artery aneurysm with intimal damage and mural
  thrombus
\item
  Stage 3: Distal embolisation from subclavian artery disease
\end{itemize}

\hypertarget{diagnosis}{%
\subsubsection{Diagnosis}\label{diagnosis}}

Most useful studies are pulse volume recordings (PVR) and duplex to
identify aneurysm or sites of embolization. Stress test is not reliable
for diagnosis. \citep{vemuriClinicalPresentationManagement2017, criadoSpectrumArterialCompression2010}

\hypertarget{management-considerations-with-atos}{%
\subsubsection{Management considerations with aTOS}\label{management-considerations-with-atos}}

\begin{itemize}
\item
  Symptomatic patients are generally indicated for treatment. Unlike
  asymptomatic patients in whom it may be appropriate to manage
  conservatively. \citep{boll122ThoracicOutlet2019}
\item
  Supraclavicular rib resection is the most suitable for adequate
  arterial reconstruction. Transaxillary has been argued to offer more
  complete rib resection however arterial repair is not possible in
  this approach.
\item
  Subclavian artery repair is necessary in Scher Stages 2 and 3 and in
  some cases 1. Arterial repair with conduit either GSV, Femoral Vein
  or prosthetic have been described. Ringed PTFE offers good patency
  and resistance to kinking in this functional anatomical location.~
\end{itemize}

\hypertarget{ntos}{%
\subsection{nTOS}\label{ntos}}

\hypertarget{demographics-1}{%
\subsubsection{Demographics}\label{demographics-1}}

Neurogenic TOS is largely a clinical diagnosis with symptoms and signs
pertaining to nerve compression most commonly the lower trunk of the
brachial plexus.~

\begin{itemize}
\item
  F\textgreater M -- 70\% Female
\item
  Ages 20-40
\item
  Occupational Exposure
\item
  Trauma history
\end{itemize}

\hypertarget{presentation-of-ntos}{%
\subsubsection{Presentation of nTOS}\label{presentation-of-ntos}}

\begin{itemize}
\item
  Symptoms \citep{sadeghi-azandaryaniThoracicOutletSyndrome2009, sandersDiagnosisThoracicOutlet2007}

  \begin{itemize}
  \item
    Paraesthesia (98\%)
  \item
    Trapezius pain (92\%)
  \item
    Neck, shoulder or arm pain (88\%)
  \item
    Supraclavicular pain with or without occipital headache (76\%)
  \item
    Chest pain (72\%)
  \item
    Weakness
  \item
    Swelling
  \end{itemize}
\item
  Positional Effects

  \begin{itemize}
  \item
    Reproducible exacerbation of symptoms
  \item
    Lying supine with arms overhead
  \item
    Overhead activities -occupational or recreational
  \end{itemize}
\item
  Weakness and Muscle Atrophy

  \begin{itemize}
  \item
    Hypothenar atrophy
  \item
    Drop-off in athletic performance
  \item
    Inability to carry out activities of daily living~
  \end{itemize}
\end{itemize}

\hypertarget{the-role-of-the-vascular-surgeon-with-ntos}{%
\subsubsection{The role of the Vascular Surgeon with nTOS}\label{the-role-of-the-vascular-surgeon-with-ntos}}

Often these patients have already seen multiple specialists and
physiotherapists.

\begin{itemize}
\item
  Exclude other causes
\item
  Confirm diagnosis -- Neurophysiologic Tests (EMG and NCS)
\item
  Seek alternate opinion
\item
  Trial of Physiotherapy and non-operative management - patients
  should be evaluated and undergo a 6 week course of physical therapy.
  This physical therapy focuses on scalene and pectoralis stretching
  improving mobility of the shoulder and strengthening the arm. Many
  improve with physical therapy.
  \citep{baldermanPhysicalTherapyManagement2019}
\item
  Anterior scalene lidocaine block may provide temporary symptom
  relief (\textasciitilde7 days) and may help identify those patients most likely
  to benefit from surgical decompression.
  \citep{salhanPC214UltrasoundGuidedAnesthetic2016, lumImpactAnteriorScalene2012}
\item
  Botulinum injection may give an average of 6 weeks of relief.
  \citep{salhanPC214UltrasoundGuidedAnesthetic2016}
\item
  Be selective in patients who may require surgery
\end{itemize}

Surgery with Rib resection often is accomplished with transaxillary or
supraclavicular approach, particularly if scalenectomy or cervical rib
resection is necessary.

\hypertarget{abdominaliliacperipheral-aneurysms}{%
\chapter{Abdominal/Iliac/Peripheral Aneurysms}\label{abdominaliliacperipheral-aneurysms}}

\textbf{30 Mar 2021:} \emph{Mia Miller, MD and Julie Duke, MD; University of
Minnesota}

\hypertarget{pathogenesis-presentation-and-risk-factors}{%
\section{Pathogenesis, presentation and risk factors}\label{pathogenesis-presentation-and-risk-factors}}

\textbf{What is an abdominal aortic aneurysm (AAA)?}
\citep{mooreVascularEndovascularSurgery2019}

\begin{itemize}
\item
  Defined as a localized dilation of an artery to a diameter greater
  than 50\% (1.5x) of its normal diameter. It is generally accepted
  that \textgreater3cm in adults is considered aneurysmal for the abdominal
  aorta.
\item
  AAAs can be described as:

  \begin{itemize}
  \item
    Infrarenal -- distal to the renal arteries with normal aorta
    between the renal arteries and the aneurysm origin.
  \item
    Juxtarenal -- aneurysm extends to the renal arteries but does
    not involve them
  \item
    Pararenal -- aneurysm involving the origin of at least one of
    the renal arteries
  \end{itemize}
\item
  Estimated 1.1 million Americans have AAAs, which equates to a
  prevalence of 1.4\% in 50-84 year old general population.
\item
  AAAs are 3-7x more prevalent than thoracic aortic aneurysms and can
  co-exist with other aneurysms throughout the arterial vascular
  system like popliteal artery aneurysms.

  \begin{itemize}
  \item
    In a 10-year review originating from Ireland, 50\% of patients
    that presented with unilateral popliteal artery aneurysms had
    associated AAA. In patients with bilateral popliteal aneurysms,
    63\% of those had associated AAA.
    \citep{duffyPoplitealAneurysms10year1998}
  \item
    Conversely, if a patient is first found to have a AAA, there is
    an 11\% chance of having associated popliteal artery
    aneurysms \textgreater15mm. \citep{tuvesonPatientsAbdominalAortic2016}
  \item
    Another study showed a rate of femoral-popliteal aneurysms in
    AAA patients is approximately 14\%.
    \citep{diwanIncidenceFemoralPopliteal2000}
  \item
    This association stresses the importance of a good physical exam
    when evaluating a patient with a AAA and is commonly tested on
    exams.
  \end{itemize}
\end{itemize}

\textbf{What is the pathogenesis of an abdominal aortic aneurysm?}
\citep{mooreVascularEndovascularSurgery2019}

\begin{itemize}
\item
  More than 90\% of AAAs are associated with atherosclerosis.
\item
  Other causes include cystic medial necrosis, dissection, Marfan's
  syndrome, Ehler's-Danlos syndrome, HIV and syphilis.
\item
  Elastin and collagen are the major structural proteins responsible
  for the integrity of the aortic wall and defects in these cause
  degeneration and further aneurysmal change.
\item
  For example, a mutation in fibrillin in Marfan's syndrome causes
  elastin fragmentation and pathological remodeling of the wall of the
  artery to form cystic medial degeneration.
\item
  Several investigations have also shown that upregulations of
  metalloproteinase activity, specifically MMP-2 and MMP-9, have an
  essential role in aneurysm formation. Imbalances between aortic wall
  proteases and antiproteases cause degradation of the extracellular
  matrix and loss of structural integrity of the aortic wall.
\item
  Increased thrombus burden is associated with wall thinning, medial
  loss of smooth muscle cells, elastin degradation, adventitial
  inflammation and aortic wall hypoxia which all increase the rate of
  AAA growth.
\end{itemize}

\textbf{What are the risk factors for AAA occurrence and growth?}
\citep{mooreVascularEndovascularSurgery2019}

\begin{itemize}
\item
  Risk factors for AAAs are similar to the risk factors for occlusive
  atherosclerosis and include age, tobacco use, hypertension, male
  gender and hypercholesterolemia.
\item
  It has been found that diabetes is protective for AAA progression
  and rupture.
\item
  Cigarette smoking is the single most important modifiable risk
  factor to prevent occurrence and growth of AAAs. Smoking increases
  the rate of growth by 35\% for abdominal aortic aneurysms.
\item
  Medical therapy has been studied with disappointing results.
  Beta-blockers and ACE/ARB inhibitors have been studied but have not
  shown any effect on growth of AAAs.
\item
  Fluoroquinolones

  \begin{itemize}
  \item
    In a recent study just published in JAMA Surgery this January,
    the group at UNC showed an increased short-term risk of
    developing an aortic aneurysm with fluoroquinolone use.
    \citep{newtonAssociationFluoroquinoloneUse2021a}
  \item
    They reviewed all prescription fills for fluoroquinolones or
    comparative antibiotics from 2005-2017.
  \item
    This included \textgreater27 million US Adults aged 18-64 years old with no
    history of aneurysms.
  \item
    18\% of the prescriptions were fluoroquinolones.
  \item
    Fluoroquinolones were associated with increased incidence of
    aortic aneurysms. Compared to the other antibiotics,
    fluoroquinolones were associated with a higher 90-day incidence
    of AAA and iliac aneurysms as well as more likely to undergo
    aneurysm repair.
  \item
    They recommended that fluoroquinolone use should be pursued with
    caution in all adults, not just high risk individuals, and they
    recommended broadening of the warnings from the FDA.
  \end{itemize}
\item
  Fluoroquinolones playing a role in dissections and aneurysm
  formation is often a highly tested question
\end{itemize}

\textbf{What is the dreaded complication of AAA?}
\citep{mooreVascularEndovascularSurgery2019}

\begin{itemize}
\item
  Aneurysm rupture is the fear with a diagnosis of AAA. The risk of
  rupture increases yearly as the aneurysm expands. Once an aneurysm
  develops, it tends to enlarge gradually yet progressively. This is
  an important concept to grasp for testing.
\item
  Growth rate

  \begin{itemize}
  \item
    For smaller aneurysms (3-5cm in size), the growth rate is
    approximately 2-3mm/year
  \item
    For larger aneurysms (\textgreater5cm), the growth rate is higher at
    3-5mm/year.
  \end{itemize}
\item
  Rupture risk (historically):

  \begin{itemize}
  \item
    4 - 5.4cm -\textgreater{} 0.5-1\%.
  \item
    6 - 7cm -\textgreater{} 10\%
  \item
    7 -- 8cm -\textgreater{} 19-35\%
  \end{itemize}
\item
  Newer data suggests the true rupture risk per year is decreasing
  with time.
\item
  In a study from the UK published in JVS in 2015, the rupture risks
  were far lower than previously reported and what is documented in
  most textbooks. \citep{parkinsonRuptureRatesUntreated2015}

  \begin{itemize}
  \item
    This systematic review of more recently published data mostly
    from 1995 to 2014 included a total of 11 studies reviewing 1514
    patients. The cumulative yearly rupture risks identified in this
    study were as follows:

    \begin{itemize}
    \item
      5.5 - 6 cm -\textgreater{} 3.5\%
    \item
      6.1 - 7 cm -\textgreater{} 4.1\%
    \item
      \textgreater7 cm -\textgreater{} 6.3\%
    \end{itemize}
  \item
    Previously published data with meta-analyses from 1970s-1990s
    reported rupture rates of 3.3\%, 9.4\% and 24\%, respectively,
    compared to 3.5\%, 4.1\% and 6.3\% in the most recent data.
  \end{itemize}
\item
  Factors that increase the risk of rupture other than the size of the
  aneurysm are smoking, COPD, hypertension, transplant recipient, and
  rapid enlargement (defined as 1\,cm/year or more).
\end{itemize}

\hypertarget{evaluation-and-diagnosis}{%
\section{Evaluation and Diagnosis}\label{evaluation-and-diagnosis}}

\textbf{What is the work up for a AAA?}
\citep{mooreVascularEndovascularSurgery2019}

\begin{itemize}
\item
  75\% of all infrarenal AAAs are asymptomatic when first detected and
  often incidentally discovered on unrelated imaging.
\item
  Symptoms - Some patients may report symptoms such as abdominal,
  flank or back pain from pressure on adjacent somatic sensory nerves
  or overlying peritoneum. Tenderness by itself is not a reliable
  indicator of impending rupture. Other symptoms include thrombosis
  and distal embolization.
\end{itemize}

\hypertarget{imaging}{%
\subsection{Imaging}\label{imaging}}

\begin{itemize}
\item
  Ultrasound, when feasible, is the preferred imaging modality for
  aneurysm screening and surveillance.

  \begin{itemize}
  \item
    The Society for Vascular Surgery (SVS) recommends a one-time
    ultrasound screening in men and women ages 65 to 75 years with
    either a history of smoking or a family history of AAA, as well
    as men and women over the age of 75 with a smoking history in
    otherwise good health who have not previously undergone
    screening. \citep{chaikofSocietyVascularSurgery2018a} Recommended
    intervals for surveillance imaging:

    \begin{itemize}
    \item
      2.5 -- 2.9 cm -\textgreater{} 10 years
    \item
      3 - 3.9 cm -\textgreater{} 3 years
    \item
      4 - 4.9 cm -\textgreater{} 1 year
    \item
      5 - 5.4 cm -\textgreater{} 6 months
    \end{itemize}
  \item
    It is important to note that these screening guidelines are
    Level 2, Grade C evidence from the SVS.
  \item
    Traditionally, once duplex reveals an aneurysm 5cm in size, an
    initial CTA is performed and patients are followed with
    additional CT scans to assist with operative planning.
  \end{itemize}
\item
  CT Angiograms are helpful in operative planning and determining
  candidacy for EVAR. You can assess the relationship of the aneurysm
  to the renal arteries, assess the access vessels, and measure seal
  zones

  \begin{itemize}
  \tightlist
  \item
    The maximum aneurysm diameter derived from the CTA should be
    based on outer wall to outer wall measurement perpendicular to
    the path of the aorta (the centerline of the aneurysm).
  \end{itemize}
\item
  MRA is recommended for patients with renal insufficiency who cannot
  tolerate iodinated contrast.
\end{itemize}

\hypertarget{management-1}{%
\section{Management}\label{management-1}}

\textbf{What are the indications for repair?}
\citep{mooreVascularEndovascularSurgery2019}

\begin{itemize}
\item
  The current recommendation to repair a fusiform aneurysm is 5.5cm
  for men (Level 1, Grade A evidence), 5.0cm for women as they have a
  higher risk for rupture, and rapid growth (\textgreater5mm over 6 months).
  \citep{chaikofSocietyVascularSurgery2018a}
\item
  For saccular aneurysms, the SVS practice guidelines recommend
  elective repair (Level 2, Grade C evidence).
  \citep{chaikofSocietyVascularSurgery2018a}

  \begin{itemize}
  \item
    Studies show equivalent wall stress in saccular aneurysms at
    much smaller sizes when compared to fusiform aneurysms. This has
    led to the notion that they have a higher rupture risk at
    smaller sizes.
  \item
    A study published in Annals of Vascular Surgery in 2016 showed a
    significant portion of ruptures \textless55mm in size were saccular in
    nature. \citep{kristmundssonMorphologySmallAbdominal2016}

    \begin{itemize}
    \tightlist
    \item
      Specific size guidelines for repair are currently lacking
      because of their infrequent presentation.
    \end{itemize}
  \end{itemize}
\end{itemize}

\textbf{What are the options for repair, and how do you choose?}
\citep{mooreVascularEndovascularSurgery2019, fairman72AortoiliacAneurysms2019}

\begin{itemize}
\item
  Two options: open repair and endovascular aortic aneurysm repair
  (EVAR).

  \begin{itemize}
  \tightlist
  \item
    When attempting to decide between the two, one must consider the
    patient's perioperative risk as well as the patient's anatomy,
    which will be reviewed further here.
  \end{itemize}
\item
  When reviewing the patient's risk for surgery, there are many tools
  to assist, which are outlined in the Society for Vascular Surgery's
  practice guidelines.
\item
  The VSGNE or Vascular Study Group of New England developed a risk
  prediction model for mortality which can assist in your decision
  making. This is endorsed by both SVS and the Vascular Quality
  Initiative.

  \begin{itemize}
  \item
    This risk model looks at open vs endovascular repair and further
    delineates infrarenal vs suprarenal clamps
  \item
    It includes aneurysm sizes with 6.5cm as the cut off.
  \item
    It includes age above or below 75yo.
  \item
    Gender and comorbidities are included like heart disease,
    cerebrovascular disease and COPD.
  \item
    An important risk factor is also renal function which is
    delineated by creatinine at 1.5-2 or \textgreater2.
  \item
    Each of these risk factors is assigned a point grading.
  \item
    These points are added together and they place the patient on a
    spectrum of mortality risk. Depending on the amount of points
    accumulated, the risk is divided into low, medium, high or
    prohibitively high-risk groups
  \item
    This is something that can help both the patient and physician
    in deciding on surgery and how to proceed.
  \end{itemize}
\item
  Recent studies have shown that decreased aerobic fitness and high
  frailty score predicted increased morbidity and mortality after open
  aneurysm repair.
\item
  High-risk patients are defined by the following in the SVS
  guidelines:

  \begin{itemize}
  \item
    Unstable angina or angina at rest
  \item
    Congestive heart failure with EF \textless{} 25-30\%
  \item
    Serum creatinine level \textgreater{} 3\,mg/dL
  \item
    Pulmonary disease manifested by room air PaO2 \textless{} 50\,mmHg,
    elevated PCO2, or both.
  \end{itemize}
\item
  To help delineate a patient's risk, a preoperative workup is
  necessary. The SVS practice guidelines recommend the following:
  \citep{chaikofSocietyVascularSurgery2018a}

  \begin{itemize}
  \item
    Determine if the patient has an active cardiovascular condition.
    Coronary artery disease is responsible for at least 50\% to 60\%
    of perioperative and late deaths after operations on the
    abdominal aorta, therefore, it is important for patients to
    undergo cardiac evaluation prior to surgery.

    \begin{itemize}
    \item
      Unstable angina, decompensated heart failure, severe
      valvular disease, significant arrythmia -\textgreater{} Cardiology
      consultation (Level 1, Grade B)
    \item
      Significant clinical risk factors such as coronary artery
      disease, congestive heart failure, stroke, diabetes
      mellitus, and chronic kidney disease -\textgreater{} Stress test (Level
      2, Grade B)
    \item
      Worsening dyspnea -\textgreater{} Echocardiogram (Level 1, Grade A)
    \item
      All patients undergoing EVAR or open repair require EKG
    \item
      In patients capable of moderate physical activities, such as
      climbing two flights of stairs or running a short distance
      (MET \textgreater= 4), there is no benefit in further testing.
    \item
      If coronary intervention is required, this takes precedence
      over aneurysm repair.
    \end{itemize}
  \item
    History of COPD

    \begin{itemize}
    \item
      Pulmonary function test with ABG (Level 2, Grade C)
    \item
      Smoking cessation for at least 2 weeks prior (Level 1,
      Grade C)
    \item
      Pulmonary bronchodilators at least 2 weeks before aneurysm
      repair (Level 2, Grade C)
    \end{itemize}
  \end{itemize}
\item
  In patients who are deemed high risk, EVAR is the most attractive
  option in anatomically suitable patients
\item
  Morbidity and mortality rates are lower for EVAR than open repair in
  the short term. This is illustrated in multiple studies.

  \begin{itemize}
  \item
    The EVAR-1 trial, a randomized prospective UK study including
    1082 patients, compared EVAR with open AAA repair in patients
    who were fit enough to undergo open surgical repair from
    1999-2003. The 30-day mortality rate was reduced in the EVAR
    group (1.7\% vs 4.7\%), although secondary interventions were more
    common in the EVAR group (9.8\% vs.~5.8\%).
    \citep{greenhalghComparisonEndovascularAneurysm2004}
  \item
    The DREAM trial, a multicenter randomized trial from 2000-2003,
    compared open repair with EVAR in 345 patients with a reduction
    in operative mortality (4.7\% vs 9.8\%) with the majority of
    complications accounted for by pulmonary issues.
    \citep{prinssenRandomizedTrialComparing2004}
  \end{itemize}
\item
  This early survival benefit with EVAR over open repair disappears by
  the third postoperative year.

  \begin{itemize}
  \item
    The Open vs Endovascular Repair (OVER) trial included 881
    patients from 42 VA centers randomized to either EVAR or open
    repair. This demonstrated that perioperative mortality was
    improved in the EVAR group (0.5\% vs 3.0\%), yet no statistically
    significant difference was seen in mortality at 2 years (7.0\% vs
    9.8\%). \citep{lederleOpenEndovascularRepair2019}
  \item
    Late mortality seems to be higher in EVAR due to ruptures from
    endoleaks that do not occur in open repair.
    \citep{rajendranLateRuptureAbdominal2017}
  \end{itemize}
\item
  Reviewing the anatomic criteria for traditional EVAR may rule out
  EVAR as an option in some patients. These criteria vary slightly
  depending on the particular device being used.

  \begin{itemize}
  \item
    Neck

    \begin{itemize}
    \item
      A neck length of at least 10-15mm from the renal arteries to
      the aneurysm start with a diameter of 18-32mm.
    \item
      It is important that the neck is relatively free of thrombus
      or calcification to decrease the risk of endoleaks.
    \item
      More complex options like fenestrated EVAR are available for
      shorter necks but will not be discussed in this review.
    \end{itemize}
  \item
    Angulation

    \begin{itemize}
    \tightlist
    \item
      Neck angulation should be \textless{} 60 degrees for current devices
    \end{itemize}
  \item
    Access vessels

    \begin{itemize}
    \tightlist
    \item
      Access vessels must be adequate for delivery of the device
      depending on the sheath size required (6-8mm)
    \end{itemize}
  \item
    Aortic bifurcation

    \begin{itemize}
    \tightlist
    \item
      The aortic bifurcation must be \textgreater20mm in size to accommodate
      the graft opening to full caliber
    \end{itemize}
  \item
    Iliac landing zone

    \begin{itemize}
    \item
      Adequate seal zone in the distal common iliac arteries of
      10-15mm in length and diameter of 7.5-25mm.
    \item
      If covering the hypogastric arteries is necessary
      unilaterally to obtain a seal, you can embolize the
      hypogastric artery (to prevent retrograde flow) and extend
      the graft into the external iliac artery.
    \item
      If this is an issue bilaterally, an iliac branch device can
      assist in maintaining perfusion into the hypogastric
      arteries.
    \end{itemize}
  \end{itemize}
\end{itemize}

\hypertarget{evar}{%
\subsection{EVAR}\label{evar}}

\textbf{Can you briefly go over the steps of an EVAR?}
\citep{mooreVascularEndovascularSurgery2019}

\begin{itemize}
\item
  EVAR now accounts for approximately 70-80\% of elective abdominal
  aortic aneurysm repairs and 65\% of iliac aneurysm repairs in the
  United States and many other countries.
\item
  Performed in the operating room or IR suite with a fixed or portable
  C-arm
\item
  Anesthesia

  \begin{itemize}
  \tightlist
  \item
    Regional block, local anesthesia or general anesthesia depending
    on surgeon preference and patient risk
  \end{itemize}
\item
  Groin access and short sheath placement

  \begin{itemize}
  \item
    Percutaneous - Closure devices are introduced prior to inserting
    the large sheaths containing the stent-grafts
  \item
    Cut-down
  \end{itemize}
\item
  Pigtail catheter is used to perform an aortogram of the abdominal
  aorta and iliac arteries
\item
  The renal artery orifices are marked. If there is any concern about
  good visualization, IVUS (intravascular ultrasound) can be used to
  assist.
\item
  Systemic heparin is given
\item
  Bilateral femoral sheaths are exchanged over a stiff wire for the
  necessary sheaths required for the device size chosen.

  \begin{itemize}
  \item
    Main trunk and ipsilateral limb sheath on one side
  \item
    Contralateral limb sheath on the other side
  \end{itemize}
\item
  The main body is positioned in the proximal neck and a repeat
  angiogram is commonly performed to confirm the positioning of the
  device at the desired level just below the lowest renal artery. It
  is best to position the main body so that the gate is directed at
  the simplest angle to cannulate.
\item
  The main body is deployed to the point where the gate is opened
\item
  Contralateral limb gate cannulation is performed using a wire and
  directional catheter.
\item
  Once in the gate, a pigtail catheter is formed within the main body
  and must be able to spin freely 360 degrees to confirm placement
  within the endograft
\item
  The contralateral limb is introduced and deployed taking care to
  preserve flow to the internal iliac artery.
\item
  The remainder of the main body is deployed and iliac extensions
  deployed if required.
\item
  The stent graft is ballooned at the neck, within the gate, at the
  bifurcation, and distal iliac seal zones.
\item
  An aortogram, usually multiple in different views, is performed to
  exclude any endoleaks.
\item
  The sheaths are removed, and the groin sites are closed using
  Perclose devices if performed percutaneously, or primary repair if
  open cutdown performed.
\item
  Check pedal signals at the end of the case to ensure no
  thromboembolic events or femoral artery access injuries have
  occurred. If there is concern, an ultrasound duplex can be performed
  intraoperatively.
\end{itemize}

\textbf{You mentioned endoleaks, can you discuss the complications specific to
EVAR and the management?} \citep{mooreVascularEndovascularSurgery2019}

\begin{itemize}
\item
  Many of the cardiopulmonary complications inherent with open repair
  do not occur with EVAR as there is no aortic cross clamping.

  \begin{itemize}
  \tightlist
  \item
    In a study from Mayo clinic evaluating elective infrarenal AAA
    repairs from 1999 to 2001, Elkouri et al found that cardiac and
    pulmonary morbidity after EVAR was drastically reduced compared
    to open repair (11\% vs 22\% and 3\% vs 16\%, respectively).
    \citep{elkouriPerioperativeComplicationsEarly2004}
  \end{itemize}
\item
  Risk of ischemic colitis remains as the IMA is covered with EVAR. It
  is lower than with open repair but remains 1-2\%.
\item
  Renal insufficiency may occur secondary to contrast administration
  in a patient with underlying chronic kidney disease. Thromboembolic
  events may occur from thrombus-laden aortic necks with wire and
  device manipulation to the renal arteries as well.
\end{itemize}

\hypertarget{endoleaks}{%
\subsubsection{Endoleaks}\label{endoleaks}}

Defined as persistent blood flow within the aneurysm sac following EVAR.

\begin{enumerate}
\def\labelenumi{\arabic{enumi}.}
\item
  Type I

  \begin{itemize}
  \item
    A leak at the graft ends secondary to inadequate seal proximally
    (1a) or distally (1b)
  \item
    If identified intraoperatively, Type I endoleaks require
    attention with further balloon angioplasty, proximal or distal
    extension, or endoanchors.
  \item
    If seen in follow up surveillance, intervention is necessary.
  \end{itemize}
\item
  Type II

  \begin{itemize}
  \item
    Sac filling secondary to retrograde filling via a branch vessel
    off of the aneurysm sac such as a lumbar artery or the IMA
  \item
    If identified intraoperatively, this typically does not need to
    be addressed in the OR.
  \item
    Typically, type II endoleaks spontaneously thrombose and
    therefore can be observed.
  \item
    If the leak persists for \textgreater{} 6 months with sac enlargement \textgreater5mm,
    intervention is recommended. Several techniques exist to
    eliminate type II endoleaks, most frequently embolization.
  \item
    It is common to continue monitoring even if there is persistent
    flow as long as there is no aneurysm sac growth.
  \end{itemize}
\item
  Type III

  \begin{itemize}
  \item
    Separation of graft components
  \item
    Usually identified in follow-up surveillance and necessitates
    intervention.
  \end{itemize}
\item
  Type IV

  \begin{itemize}
  \tightlist
  \item
    Secondary to a porous graft which typically does not occur any
    longer as endograft material and devices have improved. If seen,
    no intervention is needed at the time, and they usually
    thrombose on their own.
  \end{itemize}
\item
  Type V

  \begin{itemize}
  \item
    Increasing aneurysm sac size with no identifiable endoleak.
    Commonly referred to as endotension.
  \item
    Usually necessitates graft explantation and open repair or
    re-lining of the graft.
  \end{itemize}
\end{enumerate}

\hypertarget{open-repair}{%
\subsection{Open Repair}\label{open-repair}}

\textbf{Now we can move onto open repair. Describe an open infrarenal aneurysm
repair.} \citep{mooreVascularEndovascularSurgery2019, garygwindAnatomicExposuresVascular2013}

\begin{itemize}
\item
  After thorough preoperative evaluation and clearance, the patient is
  taken back to the operating room. An epidural may be placed
  preoperatively depending on institutional preference. The patient is
  intubated, and arterial and central venous catheters are placed.~
  The abdomen is prepped from chest to bilateral thighs.
\item
  A cell-saver should be available to optimize resuscitation during
  the procedure due to expected large amounts of blood loss. Balanced
  resuscitation to prevent coagulopathy is important with significant
  blood loss.
\item
  Exposure

  \begin{itemize}
  \tightlist
  \item
    Trans-peritoneal or retro-peritoneal. First we will describe the
    most common approach: trans-peritoneal.
  \end{itemize}
\item
  Surgical steps

  \begin{enumerate}
  \def\labelenumi{\arabic{enumi}.}
  \item
    Mid-line laparotomy, transverse or chevron-style incision
  \item
    A retractor system such as an Omni, Bookwalter or Balfour
    retractor is used to assist in exposure depending on physician
    preference.
  \item
    The transverse colon is retracted cephalad, and the small bowel
    is retracted to the patient's right to expose the aorta. The
    duodenum is mobilized and the ligament of Treitz is divided. The
    posterior peritoneum is opened along the anterior wall of the
    aorta.
  \item
    The aneurysm sac is now in view and careful dissection
    proximally for clamp site is achieved. Identification of the
    left renal vein crossing the aorta is key and can be divided if
    necessary.
  \item
    Identification of the renal arteries proximally is required if
    there is a plan for suprarenal clamping.
  \item
    Isolate bilateral common iliac arteries for distal clamp site.
    Use caution when dissecting the fibro-areolar tissue overlying
    the left common iliac artery as it contains nerves that control
    sexual function. Damage can result in retrograde ejaculation.

    \begin{itemize}
    \item
      You can avoid nerve injury with mobilization of the sigmoid
      colon medially and identifying the iliac bifurcation
      distally, thus avoiding transecting the tissue overlying the
      left common iliac artery.
    \item
      If the iliac arteries are severely calcified and pose risk
      for injury with clamping, intraluminal balloon catheters can
      be inserted for distal control instead.
    \item
      Also, you must be cognizant of the location of the ureters
      crossing over the iliac bifurcation to prevent injury.
    \end{itemize}
  \item
    After proximal and distal clamp sites have been identified,
    systemic heparin is administered by anesthesia.
  \item
    Clamp the distal vessels first to prevent distal embolization.
  \item
    Open the aneurysm sac in a longitudinal fashion toward the
    patient's right to avoid the IMA and clear the sac of thrombus.
    Extend proximally to normal aorta and then t off the incision on
    the aortic wall.

    \begin{itemize}
    \tightlist
    \item
      Some physicians prefer to transect the aortic wall as
      opposed to leaving the posterior wall intact for the
      anastomosis.~
    \end{itemize}
  \item
    Lumbar arteries on the posterior wall are ligated using
    figure-of-eight sutures.

    \begin{itemize}
    \tightlist
    \item
      Back-bleeding lumbar vessels can be the source of
      significant blood loss.
    \end{itemize}
  \item
    Graft

    \begin{itemize}
    \item
      A tube graft or bifurcated graft depending on the patient's
      anatomy and aortic diameter is chosen. Dacron or PTFE grafts
      are most common, and the choice depends on physician
      preference. This is anastomosed proximally in a continuous
      fashion.
    \item
      Once complete, the graft is flushed forward to flush out any
      thrombus. The graft is then clamped and the aortic clamp
      removed to test the anastomosis. Repair if needed.
    \item
      The distal anastomosis is completed to the aorta or
      bilateral iliac arteries depending on extent of the
      aneurysm.
    \item
      The graft is flushed forward prior to completion to remove
      any thrombus within the graft. The anastomosis is completed
      and clamps removed.~
    \end{itemize}
  \item
    Hypotension may occur at this point from re-perfusion of the
    lower extremities and pelvis. Anesthesia should be notified that
    unclamping will occur soon prior to completion of the distal
    anastomosis to allow for fluid resuscitation in preparation.

    \begin{itemize}
    \tightlist
    \item
      The graft can be slowly unclamped or partially clamped to
      assist with blood pressure management during this time. You
      can also place manual pressure on the iliac arteries or
      femoral arteries to slowly release flow and avoid
      significant hypotension.
    \end{itemize}
  \item
    Next, the IMA must be addressed. The IMA orifice is identified
    within the aneurysm sac.~

    \begin{itemize}
    \item
      Chronically occluded or pulsatile back bleeding -\textgreater{} ligate.
    \item
      Anything between occlusion and strong pulsatile back
      bleeding requires further evaluation. This should be
      performed at the end of the case after the internal iliacs
      have been reperfused. Methods to measure perfusion:

      \begin{itemize}
      \item
        Place vessel loops or micro bulldog on IMA and assess
        the sigmoid colon. If there is a poor doppler signal on
        the antimesenteric border of the sigmoid colon, the IMA
        should be reimplanted.
      \item
        Insert blunt-tip needle through the IMA orifice and pull
        vessel loop around needle to secure and connect to a
        transducer. Pressure less than 35 mmHg requires
        reimplantation. \citep{hoballahInferiorMesentericArtery2021}
      \end{itemize}
    \item
      The Carrel patch technique involves excising a circular
      button of the aortic wall around the IMA and anastomosing it
      to the graft wall.
    \item
      Newer studies have shown that IMA reimplantation does not
      eliminate the risk of ischemic colitis after open AAA
      repair. In a study out of George Washington University in DC
      published in JVS in 2019, there was still significant risk
      of ischemic colitis rates with IMA reimplantation.
      \citep{leeInferiorMesentericArtery2019}

      \begin{itemize}
      \item
        Using NSQIP data collected prospectively and studied
        retrospectively
      \item
        Out of 2397 patients undergoing AAA from 2012-2015, 135
        patients (5.6\%) had ischemic colitis.
      \item
        672 patients were evaluated further after exclusion
        criteria applied (suprarenal clamp, emergent or
        ruptured, occluded mesenteric vessels)
      \item
        Of these, 637 patients had IMA ligation, 35 had IMA
        reimplantation
      \item
        Reimplantation was associated with - More frequent
        return to the OR (20\% vs 7.2\%), Higher rates of wound
        complications (17.1\% vs 3\%), Higher rates of ischemic
        colitis (8.6\% vs 2.4\%)
      \end{itemize}
    \item
      Difficult to interpret impact of revascularization of IMA on
      ischemic colitis rates, due to selection bias, but should be
      noted that patients who require revascularization still may
      experience colon ischemia.
    \end{itemize}
  \item
    To finish, the aneurysm sac is then closed over the graft to
    protect the viscera, and the retro-peritoneum is reapproximated.
    Occasionally, a vascularized omental pedicle flap may be used to
    separate the graft from the duodenum to prevent an aorto-enteric
    fistula if the peritoneum cannot be closed securely.
  \end{enumerate}
\item
  Steps for the retro-peritoneal approach:

  \begin{itemize}
  \item
    Positioned semi-lateral with the left side up with bilateral
    groins exposed for femoral artery access. This is done in a lazy
    lateral position where the patients upper body is near complete
    lateral but the hips are rotated to the patient's left in
    attempt to keep both groins in the field in case they need to be
    accessed.
  \item
    An oblique incision extends from the left 11th intercostal space
    or tip of the 12th rib to the edge of the rectus abdominus
    muscle, through the external and internal oblique muscles,
    transversalis fascia until you are just superficial to the
    peritoneum. Using blunt finger dissection, the peritoneum is
    dissected from the abdominal wall posteriorly over the psoas
    muscle until the aorta is reached.

    \begin{itemize}
    \item
      Benefits include less postoperative ileus, less
      intraoperative hypothermia, lower IV fluid requirements, and
      less post-op respiratory compromise.
    \item
      A disadvantage is the difficulty addressing the right iliac
      artery from this approach.
    \end{itemize}
  \end{itemize}
\end{itemize}

\hypertarget{complications}{%
\subsubsection{Complications}\label{complications}}

\textbf{What are some of the complications with open aortic aneurysm repair?}
\citep{mooreVascularEndovascularSurgery2019}

\begin{itemize}
\item
  Myocardial dysfunction which is usually secondary to cardiac
  ischemia or hemorrhage.
\item
  Abdominal compartment syndrome secondary to coagulopathic bleeding
  postoperatively or third spacing of fluids can cause abdominal
  compartment syndrome requiring emergent laparotomy. Unexplained
  oliguria, difficulty maintaining adequate ventilation, and
  hypotension with significant abdominal distension is concerning for
  abdominal compartment syndrome. A sustained bladder pressure \textgreater{} 20
  mmHg with associated organ dysfunction (elevated peak airway
  pressures, new onset acute renal failure) is indicative of abdominal
  compartment syndrome.

  \begin{itemize}
  \item
    Abdominal compartment syndrome can still occur after EVAR during
    an aortic rupture, therefore, one must keep a heightened
    suspicion for this in the post-operative period.
  \item
    It is important to note that a patient with a soft abdominal
    exam can still have abdominal compartment syndrome particularly
    with an enlarged body habitus.
  \end{itemize}
\item
  Renal failure can occur due to suprarenal aortic clamping,
  atheromatous embolization or hypotension causing acute tubular
  necrosis (ATN).
\item
  Postoperative ileus is common. Duodenal obstruction from dissection
  of the ligament of Treitz can mimic a gastric outlet obstruction.
\item
  Ischemic colitis of the left colon and rectum is the most serious
  gastrointestinal complication, and the incidence ranges from 0.2 -
  10\%.

  \begin{itemize}
  \item
    3-4x more common after operations for occlusive disease than
    aneurysmal disease.
  \item
    It is important to study the collateral pathways on the
    preoperative CT scan and the patient's history to assist in
    surgical decisions regarding IMA reimplantation including:

    \begin{itemize}
    \item
      Stenosis/occlusion of the SMA
    \item
      Previous colectomy
    \item
      Hypogastric artery occlusion
    \end{itemize}
  \item
    Earliest manifestation is postoperative diarrhea, especially
    bloody diarrhea.
  \item
    Sigmoidoscopy is needed for diagnosis.

    \begin{itemize}
    \item
      Mild colon ischemia with patchy mucosal involvement should
      be treated with bowel rest, fluid resuscitation and
      antibiotics. Transmural necrosis requires emergent operation
      with colon resection. Patients can be left in discontinuity
      or an end colostomy performed depending on stability.
    \item
      The mortality rate with colon ischemia after aneurysm
      surgery is about 25\% but reaches over 50\% if bowel resection
      is required.\citep{brewsterIntestinalIschemiaComplicating1991}
      This is a very heavily tested topic for both general surgery
      and vascular surgery boards.
    \end{itemize}
  \end{itemize}
\item
  Distal ischemia from embolization downstream can lodge in larger
  vessels or cause microembolization, colloquially known as ``trash
  foot''.
\item
  Infection is rare but can be associated with graft-enteric fistula
  which is another highly tested topic.
\end{itemize}

\hypertarget{postoperative-surveillance}{%
\subsubsection{Postoperative Surveillance}\label{postoperative-surveillance}}

\textbf{What is the post-operative surveillance required for open and
endovascular approach, and how do they differ?}
\citep{mooreVascularEndovascularSurgery2019}

\begin{itemize}
\item
  That is a great question because it highlights why open repair has
  continued to be so important, especially for young, healthy
  patients.
\item
  Post-operative surveillance is necessary in the immediate
  post-operative period for open repair to evaluate incisions.
  Follow-up is only needed every 5-10 years, unless the patient
  becomes symptomatic.
\item
  In contrast, EVAR patients require a strict postoperative
  surveillance regimen to allow for detection of endoleaks, aneurysm
  sac expansion, stent fracture, limb kinking and material fatigue.

  \begin{itemize}
  \item
    CT scans at 1-, 6- and 12-month intervals initially then
    annually are recommended which raises concerns related to cost,
    cumulative radiation exposure, and contrast administration.
  \item
    Some physicians may elect to use ultrasound for surveillance
    with CTA prompted if an endoleak is identified or the sac is
    enlarged, particularly in patients with stable aneurysms.
  \item
    The long-term follow-up is often inconsistent and a study of
    19,962 Medicare beneficiaries undergoing EVAR from 2001 to 2008
    showed that 50\% of patients were lost to annual imaging
    follow-up at 5 years after
    surgery.\citep{schanzerFollowUpComplianceEndovascular2015}
  \end{itemize}
\item
  Some patients will elect for open repair to avoid frequent
  surveillance if they are a candidate for both, while other patients
  will select endovascular management to avoid the short-term effects
  like longer hospitalizations, post-operative pain, and longer
  recovery time to baseline functioning in open surgery.
\end{itemize}

\hypertarget{ruptured-aneurysms}{%
\subsection{Ruptured Aneurysms}\label{ruptured-aneurysms}}

\textbf{Although elective repair is important, can you touch on the management
of a ruptured AAA (RAAA) as our last topic of the session?}
\citep{mooreVascularEndovascularSurgery2019, lindsayChapter74Ruptured2019}

\begin{itemize}
\item
  Ruptured AAAs have declined secondary to improved medical
  management, decreased rates of smoking and superior diagnostic
  imaging and surveillance.
\item
  Traditionally, it has been taught that 50\% of ruptured AAAs die in
  the field and of those remaining, 50\% will die in the hospital. With
  time, the in-hospital mortality rate has decreased.

  \begin{itemize}
  \tightlist
  \item
    In one study out of Finland, of 712 patients with ruptured AAAs
    from 2003-2013, 52\% died prior to arrival to the hospital. Of
    those that were offered surgery, 67\% of patients were alive at
    30 days indicating a mortality rate of 33\%.
    \citep{lainePopulationbasedStudyRuptured2016}
  \end{itemize}
\item
  Diagnostic triad on presentation:

  \begin{itemize}
  \tightlist
  \item
    Pain, syncope and known or palpable AAA.
  \end{itemize}
\item
  When a ruptured AAA is suspected or diagnosed, permissive
  hypotension is key in the initial management before surgery.

  \begin{itemize}
  \item
    Allowing systolic arterial pressures of 50-70\,mmHg as long as
    the patient is mentating appropriately.
  \item
    Limits internal bleeding which further limits loss of platelets
    and clotting factors.
  \end{itemize}
\item
  Initial management involves many considerations like patient
  stability, patient's anatomy and the surgeon's experience with
  either open or endovascular repair.
\item
  Due to the developments of endovascular techniques, it is ideal to
  have a CTA prior to the operating room to determine if the patient
  is a candidate for an EVAR.
\item
  There are two options for expedient aortic control in an unstable
  patient with a ruptured aneurysm.

  \begin{itemize}
  \item
    Open supraceliac aortic clamping

    \begin{itemize}
    \item
      Achieved by retracting the stomach caudally, entering and
      dividing a portion of the gastrohepatic ligament, reaching
      under and medial to the caudate lobe, dividing the pars
      flaccida, and identifying the spine. The aorta lies to the
      patient's left of the spine and is bluntly dissected
      anteriorly and laterally for aortic clamp placement.
    \item
      Another method of supraceliac exposure and control is to
      mobilize and reflect the left lobe of the liver, sweep the
      esophagus to the patient's left, divide the right crus of
      the diaphragm and bluntly dissect both sides of the aorta
      then apply the clamp.
    \item
      A nasogastric tube can help identify the esophagus when
      placing this clamp to ensure the esophagus has been swept to
      the patient's left and protected.
    \item
      The clamp should be moved down to the desired position for
      repair (supra or infrarenal depending on anatomy of the
      aneurysm neck) to decrease ischemia time to visceral vessels
      as soon as possible.
    \end{itemize}
  \item
    Percutaneous occlusive aortic balloon

    \begin{itemize}
    \tightlist
    \item
      Gain percutaneous access and place an occlusive aortic
      balloon for stabilization in the distal thoracic aorta. This
      will require a long support sheath, usually 12fr in size, to
      prevent distal migration of the occlusive balloon.
    \end{itemize}
  \end{itemize}
\item
  EVAR has been used increasingly to treat ruptured AAAs and offers
  many theoretical advantages over open repair.
\item
  Less invasive, eliminates risk of damage to periaortic and abdominal
  structures, decreases bleeding from surgical dissection, minimizes
  hypothermia and third space losses, and lessens the requirement for
  deep anesthesia.
\item
  EVAR has been deemed superior to open repair for the treatment of
  RAAA in many studies.

  \begin{itemize}
  \item
    In a study out of UVA published in JVS in August 2020, they
    looked at ruptures in the VQI database from 2003-2018. This
    resulted in 724 pairs of open and endovascular pairs after
    propensity matching. \citep{wangEndovascularRepairRuptured2020a}

    \begin{itemize}
    \item
      There was a clear advantage of endovascular compared to open
      repair in patient's with suitable anatomy.
    \item
      Length of stay was decreased with 5 vs 10 days in open. 30
      day mortality was much lower at 18\% vs 32\%. Major adverse
      events like MI, Renal failure, leg ischemia, mesenteric
      ischemia, respiratory complications were much lower in the
      EVAR group at 35\% vs 68\% in the open group.
    \item
      All cause 1 year survival was much higher with EVAR at 73\%
      vs 59\% in the open group.
    \end{itemize}
  \end{itemize}
\item
  Despite improved RAAA results with EVAR, conversion from EVAR to
  open AAA repair appears to have the most unfavorable outcomes in
  terms of mortality.

  \begin{itemize}
  \item
    Conversions can be early or late and are due to access-related
    problems, errors in endograft deployment, graft migration,
    persistent endoleak, graft thrombosis, or infection.
  \item
    In a study evaluating 32,164 patients from NSQIP with 300
    conversions (7,188 standard open repairs and 24,676 EVARs),
    conversion to open repair was associated with a significantly
    higher 30-day mortality than standard open repair (10\% vs 4.2\%)
    and EVAR (10\% vs 1.7\%). In addition, conversion patients
    compared to standard open patients were more likely to undergo
    new dialysis (6.0\% vs.~3.5\%), cardiopulmonary resuscitation
    (5.3\% vs.~1.9\%), postoperative blood transfusion (42.3\% vs.
    31.6\%), and have a myocardial infarction (5.0\% vs.~2.2\%).
    \citep{ulteeConversionEVAROpen2016}
  \end{itemize}
\end{itemize}

\hypertarget{lower-extremity-occlusive-disease}{%
\chapter{Lower Extremity Occlusive Disease}\label{lower-extremity-occlusive-disease}}

\hypertarget{section}{%
\section{}\label{section}}

\hypertarget{mesenteric-disease}{%
\chapter{Mesenteric Disease}\label{mesenteric-disease}}

\hypertarget{section-1}{%
\section{}\label{section-1}}

\hypertarget{renal}{%
\chapter{Renal}\label{renal}}

\hypertarget{section-2}{%
\section{}\label{section-2}}

\hypertarget{thoracic-aorta}{%
\chapter{Thoracic Aorta}\label{thoracic-aorta}}

\hypertarget{section-3}{%
\section{}\label{section-3}}

\hypertarget{venous-disease}{%
\chapter{Venous Disease}\label{venous-disease}}

\hypertarget{section-4}{%
\section{}\label{section-4}}

\hypertarget{vascular-trauma}{%
\chapter{Vascular Trauma}\label{vascular-trauma}}

\hypertarget{section-5}{%
\section{}\label{section-5}}

\hypertarget{angioaccess}{%
\chapter{Angioaccess}\label{angioaccess}}

\hypertarget{section-6}{%
\section{}\label{section-6}}

\hypertarget{complications-1}{%
\chapter{Complications}\label{complications-1}}

\hypertarget{section-7}{%
\section{}\label{section-7}}

\hypertarget{amputations}{%
\chapter{Amputations}\label{amputations}}

\hypertarget{section-8}{%
\section{}\label{section-8}}

\hypertarget{vascular-lab}{%
\chapter{Vascular Lab}\label{vascular-lab}}

\hypertarget{section-9}{%
\section{}\label{section-9}}

\hypertarget{vascular-medicine}{%
\chapter{Vascular Medicine}\label{vascular-medicine}}

\hypertarget{section-10}{%
\section{}\label{section-10}}

\hypertarget{endovascular}{%
\chapter{Endovascular}\label{endovascular}}

\hypertarget{section-11}{%
\section{}\label{section-11}}

\hypertarget{applied-science}{%
\chapter{Applied Science}\label{applied-science}}

\hypertarget{section-12}{%
\section{}\label{section-12}}

  \bibliography{references.bib}

\end{document}
