% Options for packages loaded elsewhere
\PassOptionsToPackage{unicode}{hyperref}
\PassOptionsToPackage{hyphens}{url}
%
\documentclass[
]{book}
\title{Vascular Surgery Board Review}
\author{Audible Bleeding}
\date{2022-01-29}

\usepackage{amsmath,amssymb}
\usepackage{lmodern}
\usepackage{iftex}
\ifPDFTeX
  \usepackage[T1]{fontenc}
  \usepackage[utf8]{inputenc}
  \usepackage{textcomp} % provide euro and other symbols
\else % if luatex or xetex
  \usepackage{unicode-math}
  \defaultfontfeatures{Scale=MatchLowercase}
  \defaultfontfeatures[\rmfamily]{Ligatures=TeX,Scale=1}
\fi
% Use upquote if available, for straight quotes in verbatim environments
\IfFileExists{upquote.sty}{\usepackage{upquote}}{}
\IfFileExists{microtype.sty}{% use microtype if available
  \usepackage[]{microtype}
  \UseMicrotypeSet[protrusion]{basicmath} % disable protrusion for tt fonts
}{}
\makeatletter
\@ifundefined{KOMAClassName}{% if non-KOMA class
  \IfFileExists{parskip.sty}{%
    \usepackage{parskip}
  }{% else
    \setlength{\parindent}{0pt}
    \setlength{\parskip}{6pt plus 2pt minus 1pt}}
}{% if KOMA class
  \KOMAoptions{parskip=half}}
\makeatother
\usepackage{xcolor}
\IfFileExists{xurl.sty}{\usepackage{xurl}}{} % add URL line breaks if available
\IfFileExists{bookmark.sty}{\usepackage{bookmark}}{\usepackage{hyperref}}
\hypersetup{
  pdftitle={Vascular Surgery Board Review},
  pdfauthor={Audible Bleeding},
  hidelinks,
  pdfcreator={LaTeX via pandoc}}
\urlstyle{same} % disable monospaced font for URLs
\usepackage{longtable,booktabs,array}
\usepackage{calc} % for calculating minipage widths
% Correct order of tables after \paragraph or \subparagraph
\usepackage{etoolbox}
\makeatletter
\patchcmd\longtable{\par}{\if@noskipsec\mbox{}\fi\par}{}{}
\makeatother
% Allow footnotes in longtable head/foot
\IfFileExists{footnotehyper.sty}{\usepackage{footnotehyper}}{\usepackage{footnote}}
\makesavenoteenv{longtable}
\usepackage{graphicx}
\makeatletter
\def\maxwidth{\ifdim\Gin@nat@width>\linewidth\linewidth\else\Gin@nat@width\fi}
\def\maxheight{\ifdim\Gin@nat@height>\textheight\textheight\else\Gin@nat@height\fi}
\makeatother
% Scale images if necessary, so that they will not overflow the page
% margins by default, and it is still possible to overwrite the defaults
% using explicit options in \includegraphics[width, height, ...]{}
\setkeys{Gin}{width=\maxwidth,height=\maxheight,keepaspectratio}
% Set default figure placement to htbp
\makeatletter
\def\fps@figure{htbp}
\makeatother
\setlength{\emergencystretch}{3em} % prevent overfull lines
\providecommand{\tightlist}{%
  \setlength{\itemsep}{0pt}\setlength{\parskip}{0pt}}
\setcounter{secnumdepth}{5}
\usepackage{booktabs}
\ifLuaTeX
  \usepackage{selnolig}  % disable illegal ligatures
\fi
\usepackage[]{natbib}
\bibliographystyle{plainnat}

\begin{document}
\maketitle

{
\setcounter{tocdepth}{1}
\tableofcontents
}
\hypertarget{about}{%
\chapter{About}\label{about}}

The content was developed here by the Audible Bleeding team to accompany our board review podcast episodes.

\hypertarget{usage}{%
\section{Usage}\label{usage}}

This is not a comprehensive guide but instead an outline of the most high yield information to help guide board preparation.

\hypertarget{comments-questions-or-contributions}{%
\section{Comments, Questions or Contributions}\label{comments-questions-or-contributions}}

Please visit our \href{https://github.com/adam-mdmph/VS-Board-Review}{github page} or \href{mailto:audiblebleeding@vascularsociety.org}{send us an emial}.

\hypertarget{cerebrovascular}{%
\chapter{Cerebrovascular}\label{cerebrovascular}}

\textbf{07 Jan 2019:} \emph{Adam Johnson, MD, MPH; Nicole Rich, MD, MPH; Kevin
Kniery, MD, MPH}

\hypertarget{available-guidelines}{%
\section{Available Guidelines}\label{available-guidelines}}

\href{https://www.jvascsurg.org/article/S0741-5214(21)00893-4/fulltext}{Society for Vascular Surgery clinical practice guidelines for
management of extracranial cerebrovascular
disease}
\citet{aburahmaSocietyVascularSurgery2022}

\hypertarget{presentation-and-diagnosis}{%
\section{Presentation and Diagnosis}\label{presentation-and-diagnosis}}

\begin{enumerate}
\def\labelenumi{\arabic{enumi}.}
\item
  \textbf{What is the definition of crescendo TIAs?}

  Frequent repetitive neurological attacks without complete resolution
  of the deficit between the episodes, producing the same deficit but
  no progressive deterioration in neurological function If a
  progressive deterioration then it is a stroke in evolution.
\item
  \textbf{Who needs to be screened?}

  Only 15\% of stroke victims have a warning TIA before a stroke so
  waiting until symptoms occur is not ideal. The purpose of carotid
  bifurcation imaging is to detect ``stroke-prone'' carotid bifurcation
  plaque and identify a high-risk patient likely to benefit from
  therapy designed to reduce stroke risk.

  The absence of a neck bruit does not exclude the possibility of a
  significant carotid bifurcation lesion - focal ipsilateral carotid
  bruits in symptomatic patients has a sensitivity of 63\% and a
  specificity of 61\% for high-grade carotid stenosis (range, 70\%-99\%).

  Screening of the general population is not indicated. Screening
  should be considered for patients with:

  \begin{itemize}
  \item
    Evidence of clinically significant peripheral vascular disease
    regardless of age
  \item
    Patients aged \textgreater65 years with a history of one or more of the
    following atherosclerotic risk factors:

    \begin{itemize}
    \item
      CAD
    \item
      Smoking
    \item
      Hypercholesterolemia
    \end{itemize}
  \item
    In general, the more risk factors present, the higher the yield
    of screening should be expected.
  \item
    The benefit of prophylactic treatment of high grade stenosis is
    estimated at a 1-2\% stroke reduction risk per year.
    \citet{naylorWhyManagementAsymptomatic2015}
  \item
    Keep in mind that intervention (CEA/CAS) has only demonstrated a
    benefit in asymptomatic patient with life expectancy greater
    than 3 years. \citep{bulbuliaAsymptomaticCarotidSurgery2017, halliday10yearStrokePrevention2010, rosenfieldRandomizedTrialStent2016}
  \end{itemize}
\item
  \textbf{US findings that confirm disease}

  \begin{itemize}
  \item
    50-69\% stenosis of ICA - Low sensitivity for 50-69\% stenosis - a
    negative ultrasound in symptomatic patients necessitates
    additional imaging

    \begin{itemize}
    \tightlist
    \item
      PSV 125-229 cm/sec
    \item
      EDV 40-100
    \item
      Internal/Common Carotid PSV Ratio 2-4
    \end{itemize}
  \item
    70-99\% stenosis of ICA

    \begin{itemize}
    \item
      PSV \textgreater/= 230 cm/sec
    \item
      EDV \textgreater100 (EDV \textgreater{} 140 cm/sec most sensitive for stenosis \textgreater80\%)
    \item
      Internal/Common Carotid PSV Ratio \textgreater{} 4
    \end{itemize}
  \item
    Velocity-based estimation of carotid artery stenosis may need to
    be adjusted in certain circumstances

    \begin{itemize}
    \item
      Higher velocities in women than in men
    \item
      Higher velocities in the presence of contralateral carotid
      artery occlusion.
    \end{itemize}
  \item
    High carotid bifurcation, severe arterial tortuosity, extensive
    vascular calcification, and obesity may also reduce the accuracy
    of DUS imaging
  \end{itemize}
\item
  \textbf{Other Imaging Modalities}

  \begin{itemize}
  \item
    CTA

    \begin{itemize}
    \item
      Pro - fast, sub-millimeter spatial resolution, visualize
      surrounding structures
    \item
      Con - cost, contrast exposure
    \end{itemize}
  \item
    MRA

    \begin{itemize}
    \item
      Pro - no contrast administered; analyze plaque morphology
    \item
      Con - Does not visualize calcium in plaque; overestimates
      the degree of stenosis (False positive for 50-69\% to be read
      as \textgreater70\%)
    \end{itemize}
  \item
    Catheter-based digital subtraction imaging (DSA)

    \begin{itemize}
    \item
      Still considered by many the gold-standard imaging modality
    \item
      Reserved for individuals with conflicting less-invasive
      imaging or those considered for CAS
    \item
      Con - cost and risk of stroke
    \end{itemize}
  \end{itemize}
\end{enumerate}

\hypertarget{management}{%
\section{Management}\label{management}}

\hypertarget{optimal-medical-therapy}{%
\subsection{\texorpdfstring{\textbf{Optimal medical therapy}}{Optimal medical therapy}}\label{optimal-medical-therapy}}

\textbf{Hypertension}

\begin{itemize}
\item
  Lowering blood pressure to a target \textless140/90 mmHg by lifestyle
  interventions and anti-hypertensive treatment is recommended in
  individuals who have hypertension with asymptomatic carotid
  atherosclerosis or those with TIA or stroke after the hyper-acute
  period.
\item
  Each 10-mm Hg reduction in blood pressure among hypertensive
  patients decreases the risk for stroke by 33\%.
\end{itemize}

\textbf{Diabetes}

\begin{itemize}
\tightlist
\item
  Glucose control to nearly normoglycemic levels (target hemoglobin
  A1C \textless7\%) is recommended among diabetic patients to reduce
  microvascular complications and, with lesser certainty,
  macrovascular complications other than stroke.
\end{itemize}

\textbf{Lipid abnormalities}

\begin{itemize}
\item
  Risk of stroke decreased by \textgreater15\% for every 10\% reduction in serum
  LDL in patients with known coronary or other atherosclerosis
\item
  Statin agents are recommended targeting LDL of 100 mg/dL, for those
  with coronary heart disease or symptomatic atherosclerotic disease,
  and LDL of 70 mg/dL for very high-risk persons with multiple risk
  factors
\item
  High dose statin therapy in patients with TIA/stroke reduce future
  rates of stroke or cardiovascular events but not overall mortality
  at 5 years. \citet{karamHighDoseAtorvastatinStroke2008}
\end{itemize}

\textbf{Smoking} - Physician counseling is an important and effective
intervention that reduces smoking in patients by 10\% to 20\%

\textbf{Antithrombotic therapy} - There is no evidence to suggest that
antiplatelet agents other than aspirin have improved benefit in
asymptomatic patients with carotid atherosclerosis

\hypertarget{carotid-endarterectomy}{%
\subsection{\texorpdfstring{\textbf{Carotid endarterectomy}}{Carotid endarterectomy}}\label{carotid-endarterectomy}}

\textbf{Timing}

\begin{itemize}
\item
  Recommendations on when to operate after a stroke

  \begin{itemize}
  \item
    Acute stroke with a fixed neurologic deficit of \textgreater6h duration -
    When the patient is medically stable, treatment in less than or
    equal to 2 weeks after the stroke is preferable.
    \citep{rothwellEndarterectomySymptomaticCarotid2004, meershoekTimingCarotidIntervention2018}
  \item
    Consider urgent intervention in a medically stable patient with
    mild-moderate neurologic deficit, if there is a significant area
    of ischemic penumbra at risk for progression
  \item
    Stroke in evolution (fluctuating / evolving neuro deficit) or
    crescendo TIA (repetitive transient ischemia w improvement
    between events) - If neuro status is not stabilized by medical
    intervention consider urgent CEA
  \item
    CEA is preferred to CAS based on an increased embolic potential
    of carotid lesions that present in this fashion.
    \citet{rantnerEarlyEndarterectomyCarries}
  \item
    Management of acute stroke \citet{powers2018GuidelinesEarly2018}

    \begin{itemize}
    \item
      \textless4.5hrs from onset of symptoms - tPA unless
      contraindication

      \begin{itemize}
      \item
        Age \textgreater80 and diabetes are contraindication to tPA after
        3hrs.
      \item
        Other contraindications - high BP, intracranial
        hemorrhage, recent stroke or head trauma, spine/brain
        surgery within 3mo, GI bleed within 21d
      \end{itemize}
    \item
      \textless6hr from onset of symptoms - catheter directed therapy
    \end{itemize}
  \end{itemize}
\item
  What is the only emergent indication for CEA?

  \begin{itemize}
  \tightlist
  \item
    Crescendo TIAs or a stroke in evolution with a surgically
    correctable lesion that is identified
  \end{itemize}
\end{itemize}

\textbf{Intraoperative Techniques}

\begin{itemize}
\item
  General concepts

  \begin{itemize}
  \tightlist
  \item
    Patch angioplasty or eversion endarterectomy are recommended
    rather than primary closure to reduce the early and late
    complications of CEA (GRADE 1, Level of Evidence A).
  \end{itemize}
\item
  Neuromonitoring/Shunting options during a carotid endarterectomy

  \begin{itemize}
  \item
    Local anesthesia with direct neuro monitoring - the patient is
    awake and moving to command throughout the case. Though improved
    neuromonitoring has not been shown to reduce MI rate with CEA
  \item
    Stump pressure Clamp the inflow and place butterfly attached to
    a-line tubing into the internal carotid If stump pressure is \textgreater{}
    40 mmHg can proceed, if \textless{} 40 place shunt
  \item
    EEG Neuromonitoring - EEG tech places neuromonitoring, monitored
    by intraop tech and neurologist remotely, generally clamp ICA
    for 3 minutes before proceeding, if any deficits unclamp, await
    normalization of EEG then proceed
  \item
    Non-selective shunting - shunt all carotids
  \end{itemize}
\item
  Techniques to reach internal carotid lesions that are high?

  \begin{itemize}
  \item
    Nasotracheal intubation will help extend the neck to reach
    higher lesions
  \item
    Divide posterior belly of digastric to reach high lesions with
    care to watch for glossopharyngeal
  \item
    Styloidectomy
  \item
    Mandible subluxation with assistance from ENT if previous
    techniques fail.
  \end{itemize}
\item
  What is the best technique for a patient with a kinked internal
  carotid artery?

  \begin{itemize}
  \item
    Eversion carotid endarterectomy will allow you to reduce the
    redundancy
  \item
    Otherwise, no advantage has been shown between eversion or
    patch, both can be shunted
  \end{itemize}
\item
  Discuss nerve injuries -- where you would encounter these and what
  deficit would be seen

  \begin{itemize}
  \item
    Hypoglossal Just above the bifurcation of the carotid artery
    Will see tongue deviation to the side of injury
  \item
    Glossopharyngeal High dissections under digastric Difficulty
    swallowing, aspiration risk, can be devastating
  \item
    Vagus Adjacent and lateral to carotid, injury occurs with
    carotid clamping, Hoarseness is noted as RLN is a branch off of
    vagus
  \item
    Marginal Mandibular (Off of facial nerve) Retraction at the
    angle of the jaw for high dissections Leads to the corner of lip
    drooping, can be confused with a neuro deficit following the
    case
  \end{itemize}
\end{itemize}

\textbf{Postoperative Complications}

\begin{itemize}
\item
  What to do if neuro deficits following your carotid endarterectomy

  \begin{itemize}
  \item
    If in OR -- perform duplex, if normal open wound and shoot
    cerebral angiogram
  \item
    If in Recovery or on the floor -- many would consider CTA first
    vs duplex to look for thrombosis
  \end{itemize}
\item
  Risk factors and how to manage hyperperfusion syndrome?

  \begin{itemize}
  \item
    Defined as an ipsilateral headache, hypertension, seizures, and
    focal neurological deficits can present 2-3 days out from
    surgery
  \item
    Patients with uncontrolled hypertension are at risk for
    hyperperfusion syndrome, clinical practice guidelines by SVS
    recommend strict BP control following CEA, maintain a pressure
    less than 140/80
  \end{itemize}
\item
  High risk groups

  \begin{itemize}
  \tightlist
  \item
    ESRD patients have higher rates of perioperative stroke, but
    also have higher rates of stroke if not revascularized.
    \citet{klarinPerioperativeLongtermImpact2016}
  \end{itemize}
\end{itemize}

\textbf{Long term complications and follow up}

\begin{itemize}
\item
  Recommend f/u US at \textless/=30 days. \textgreater/= 50\% stenosis requires further
  imaging.
\item
  Contralateral stenosis

  \begin{itemize}
  \item
    The risk of progression for moderate stenosis at the initial
    surveillance to severe stenosis can be as high as five times
  \item
    Requires post-operative surveillance.
  \end{itemize}
\end{itemize}

\hypertarget{carotid-artery-stenting}{%
\subsection{Carotid Artery Stenting}\label{carotid-artery-stenting}}

\begin{itemize}
\item
  In patients aged \textgreater70 undergoing CAS the risk of stroke was the
  highest, presumably due to calcific disease in the arch

  \begin{itemize}
  \item
    Lesion-specific characteristics are thought to increase the risk
    of cerebral vascular events after CAS and include a ``soft''
    lipid-rich plaque identified on noninvasive imaging, extensive
    (15 mm or more) disease, a pre-occlusive lesion, and
    circumferential heavy calcification
  \item
    This can be reduced, but not eliminated, by using flow-reversal
    embolic protection rather than distal filter protection
  \end{itemize}
\item
  Limited data on CAS in asymptomatic patients - currently is not
  supported by guidelines or considered reimbursable
\item
  Consider CAS in symptomatic patients with \textgreater50\% stenosis who are poor
  candidates for CEA due to severe uncorrectable medical comorbidities
  and/or anatomic considerations

  \begin{itemize}
  \item
    Ipsilateral neck dissection or XRT - equivalent periprocedural
    stroke rate to CEA, but increased later stroke rate. CEA higher
    rates of cranial nerve damage (9\%).
    \citet{giannopoulosRevascularizationRadiationinducedCarotid2018}
  \item
    Contralateral vocal cord paralysis
  \item
    Lesions that extend proximally to the clavicle or distal to C2
  \end{itemize}
\item
  Transfemoral Approach vs Transcarotid approach

  \begin{itemize}
  \tightlist
  \item
    ROADSTER Trial - single arm study with flow reversal for
    cerebral protection. Suggest lower rates of post-op stroke
  \end{itemize}
\item
  Post-op follow up - Dual-platelet therapy should be continued for 1
  month after the procedure, and aspirin should be continued
  indefinitely

  \begin{itemize}
  \tightlist
  \item
    In stent restenosis (\textgreater50\%) - repeat angioplasty or stent have
    low incidence of periprocedural stroke but failed to improve
    long term stroke/death/MI or patency rates.
    \citet{chungPercutaneousInterventionCarotid2016a}
  \end{itemize}
\end{itemize}

\hypertarget{management-of-uncommon-disease-presentations}{%
\subsection{Management of uncommon disease presentations}\label{management-of-uncommon-disease-presentations}}

\begin{itemize}
\item
  Occluded Carotid What to do for occluded carotid?

  \begin{itemize}
  \tightlist
  \item
    Leave it alone
  \end{itemize}
\item
  What if occluded carotid is still causing TIAs?

  \begin{itemize}
  \item
    External carotid endarterectomy and ligation of internal
  \item
    The addition of oral anticoagulation is likely to reduce the
    rate of recurrent CVA
  \end{itemize}
\item
  What if the patient has severe vertebrobasilar insufficiency and
  carotid artery disease?

  \begin{itemize}
  \tightlist
  \item
    Should undergo carotid revascularization first to improve flow
  \item
    Vertebrobasilar insufficiency characterized by dizziness,
    ataxia, nausea, vertigo and bilateral weakness.
    \citet{limanetoPathophysiologyDiagnosisVertebrobasilar2017}
  \end{itemize}
\item
  What about tandem lesions in the carotid in a symptomatic patient,
  carotid bulb and carotid siphon lesion (high ICA)? How should you
  treat this?

  \begin{itemize}
  \tightlist
  \item
    Treat carotid bulb first, likely the embolic source
  \end{itemize}
\item
  Carotid artery dissection

  \begin{itemize}
  \item
    Patients with carotid dissection should be initially treated
    with antithrombotic therapy (antiplatelet agents or
    anticoagulation) (GRADE 1, Level of Evidence C).
  \item
    Indications for endovascular treatment of carotid artery
    dissection \citep{cohenSinglecenterExperienceEndovascular2012, markusAntiplateletTherapyVs2019a, phamEndovascularStentingExtracranial2011}

    \begin{itemize}
    \item
      Ongoing symptoms on best medical therapy
    \item
      Contraindication to antithrombotics
    \item
      Pseudoaneurysm
    \end{itemize}
  \end{itemize}
\item
  Simultaneous coronary and carotid disease

  \begin{itemize}
  \item
    Patients with symptomatic carotid stenosis will benefit from CEA
    before or concomitant with CABG. The timing of the intervention
    depends on the clinical presentation and institutional
    experience (GRADE 1, Level of Evidence B).
  \item
    Patients with severe bilateral asymptomatic carotid stenosis,
    including stenosis and contralateral occlusion, should be
    considered for CEA before or concomitant with CABG (GRADE 2,
    Level of Evidence B)
  \item
    Patients undergoing simultaneous CEA/CABG demonstrate highest
    mortality. \citet{naylorSystematicReviewOutcomes2003}
  \end{itemize}
\end{itemize}

\hypertarget{prospective-trials---must-reads}{%
\section{Prospective Trials - MUST READS}\label{prospective-trials---must-reads}}

\begin{enumerate}
\def\labelenumi{\arabic{enumi}.}
\item
  Asymptomatic Carotid Atherosclerosis Study (ACAS)

  \begin{itemize}
  \item
    Compared medical management with CEA in asymptomatic patients
    with \textgreater{} 60\% stenosis
  \item
    5-year stroke and death rate was 5.1\% vs 11\%
  \item
    In women, the benefit of CEA was not as certain as 5y stroke and
    death rates were 7.3\% vs.~8.7\%
  \item
    This was pre statin and clopidogrel era
  \end{itemize}
\item
  North American Symptomatic Carotid Endarterectomy Trial (NASCET)
  \citet{northamericansymptomaticcarotidendarterectomytrialcollaboratorsBeneficialEffectCarotid1991}

  \begin{itemize}
  \item
    Compared medical management vs CEA for symptomatic patients with
    moderate (50-69\%) and severe stenosis (\textgreater70\%)
  \item
    Only moderate impact for patients with moderate stenosis
    (50-69\%)
  \item
    Symptomatic patients with \textgreater70 \% stenosis benefited from CEA, at
    18 months 7\% major stroke in surgical arm, and a 24\% stroke rate
    in medical arm. 29\% reduction in 5-year risk of stroke or death

    \begin{itemize}
    \tightlist
    \item
      Patients with severe \textgreater70\% stenosis had such a dramatic
      effect the trial was stopped early for this subset and all
      referred for endarterectomy
    \end{itemize}
  \item
    No benefit is shown in symptomatic patients with \textless{} 50\% stenosis
  \item
    European studies have shown similar results

    \begin{itemize}
    \item
      ACST = ACAS
    \item
      ECST = NASCET.
    \end{itemize}
  \end{itemize}
\item
  Carotid Revascularization Endarterectomy versus Stenting Trial
  (CREST)

  \begin{itemize}
  \item
    Compared CEA vs.~CAS in both symptomatic and asymptomatic
    patients.
  \item
    Composite endpoint of 30-day stroke, MI, death equivalent
    between CEA and CAS
  \item
    CAS had a significantly higher incidence of stroke and death
    than CEA and CEA higher incidence of MI

    \begin{itemize}
    \tightlist
    \item
      Follow up at 10 years demonstrated no difference in
      composite stroke/MI/death but increased rate of stroke/death
      in stented patients likely attributable to increased
      periprocedural stroke. \citet{brottLongTermResultsStenting2016b}
    \end{itemize}
  \item
    Subanalyses identified that older patients (\textgreater70y) had better
    outcomes after CEA than CAS, the QOL impact of stroke was more
    significant than that of MI, and anatomic characteristics of
    carotid lesions (longer, sequential, remote) were predictive of
    increased stroke and death after CAS
  \item
    Unfortunately, this study provides a benchmark to strive for,
    but no other large trials have achieved these results.
  \end{itemize}
\item
  ROADSTER

  \begin{itemize}
  \item
    Single arm feasibility trial of transcarotid carotid stenting
  \item
    The results of the ROADSTER trial demonstrate that the use of
    the ENROUTE Transcarotid NPS is safe and effective at preventing
    stroke during CAS. The overall stroke rate of 1.4\% is the lowest
    reported to date for any prospective, multicenter clinical trial
    of CAS.
  \end{itemize}
\item
  Trials to look out for in the next few years

  \begin{itemize}
  \item
    CREST-2 - multicenter, randomized controlled trial is underway
    that is evaluating revascularization against modern intensive
    medical management
  \item
    ACT-1 and ACST-2- the role of intervention in asymptomatic
    patients, designed to compare the early and long-term results of
    CEA vs CAS and best medical management
  \item
    ROADSTER-2 - TCAR
  \end{itemize}
\end{enumerate}

  \bibliography{references.bib}

\end{document}
